\chapter{Implantaci�n}

En este capitulo se encuentran descrito todos los pasos
y configuraciones necesarias para poder dejar la aplicaci�n
funcionando correctamente en un entorno de producci�n real.

\section{Software Necesario}
El software necesario para desplegar la aplicaci�n es el siguiente:

\begin{itemize}
	\item PostgreSQL 9.0 o superior, este es el motor de base de datos y puede
	ser descargado desde http://www.postgresql.org/download/.
	\item GlassFish 3.1.2 o superior, este es el servidor de aplicaciones 
	donde se desplegara posteriormente la aplicaci�n. Los archivos de 
	instalaci�n los puede encontrar en 
	http://glassfish.java.net/downloads/3.1.2-final.html.
\end{itemize}

La instalaci�n de las aplicaciones reci�n mencionadas var�an dependiendo del
sistema operativo donde desea instalarlo. Y en este capitulo solo se describen
configuraciones particulares necesarias para que la aplicacion funcione 
correctamente y no instalaciones genericas de estas cuya documentaci�n 
abunda en Internet.

\section{Consideraciones Previas}

\begin{itemize}
	\item No se requiere de alg�n sistema operativo en particular. 
	La �nica condici�n es que soporte Java.
	\item No es necesario que instale el servidor de aplicaciones y el servidor
	de base de datos en la misma maquina.
	\item Todos los archivos necesarios para la instalaci�n los puede 
	encontrar en la carpeta "ArchivosInstalacion" que viene dentro
	del CD que acompa�a el documento.
\end{itemize}

\section{Configurar la Base de Datos}

	\subsection{Creaci�n de la Base de Datos}
	
	Suponiendo que instalo correctamente PostgreSQL, lo siguiente es ingresar
	como administrador desde linea de comando o desde alg�n cliente
	de PostgreSQL de su preferencia ( por ejemplo PgAdmin ). Una vez dentro
	cree una nueva base de datos con el comando: 
	"CREATE DATABASE SISTEMA_SOLICITUDES" donde SISTEMA_SOLICITUDES
	es el nombre asignado a la base de datos, si desea puede utilizar otro
	nombre pero debe ser consistente con el nombre de base de datos utilizado
	en los pasos siguientes.\\

	Por ultimo seleccione la BD recien creada y ejecute el script contenido
	en el archivo "script_bd_postgres.sql" el cual contiene todas las 
	sentencias necesarias para crear la base de datos.

	\subsection{Poblamiento de la Base de Datos}
	
	Una vez creada la base de datos, es necesario cargar algunos
	datos iniciales para que la aplicaci�n trabaje correctamente.
	Dichos datos se encuentran en el archivo "datos_iniciales_postgres.sql",
	estos est�n preparados dentro de sentencias inserts por lo cual 
	basta con ejecutarlas directamente.

	\subsection{Configurar PostgreSQL}
	
	Aunque la base de datos del sistema ya esta creada y poblada,
	para que el sistema pueda interactuar con la base de datos de manera
	optima, es necesario editar algunos de los archivos de configuraci�n
	de PostgreSQL. Los archivos que debe editar son:

	\begin{itemize}
		\item \textbf{pg_hba.conf:} Este fichero se utiliza para definir los diferentes 
		tipos de accesos que un usuario tiene en el cluster. Aqu� es necesario agregar
		la IP del servidor de aplicaciones en el caso de que este en diferentes maquinas.
		Las lineas de configuraci�n deben tener el siguiente formato:\\
		[Tipo de conexion][database][usuario][IP][Netmask][Tipo de autentificacion][opciones]

		\item \textbf{postgresql.conf:} En este fichero podemos cambiar todos los 
		par�metros de configuraci�n que afectan al funcionamiento y al 
		comportamiento de PostgreSQL en nuestra maquina. Aqui lo importante
		es editar el par�metro max_connections, el cual define la cantidad m�xima de 
		clientes conectados simult�neamente. Este debe ser mayor al maximo
		de conexiones configuradas en el pool de conexiones que se crea mas adelante.
		Dado que de ser menor glassfish intentara establecer mas conexiones de las
		que la base de datos puede soportar.
	\end{itemize}

	Se puede encontrar mas informaci�n sobre como configurar estos archivos en
	http://www.postgresql.org.es/node/219 .

	\subsection{Crear Pools de conexi�n en GlassFish}
	Para que la aplicaci�n pueda establecer una conexi�n con la base de datos
	es necesario, crear un pool de conexiones en glassfish
	....

\section{Configuraci�n de Hibernate}

\section{Configuraci�n para el env�o de Correo Electronico}

\section{Configuraci�n para subir y descargar archivos}

\section{Configuraci�n de Seguridad OpenAM}