\chapter{Marco Conceptual}

\section{Conceptos y Terminologia}
	\begin{itemize}
		\item{\textbf{Solicitud:}}
		\item{\textbf{Requerimiento:}} El Glosario de Terminolog�a Est�ndar de Ingenier�a de Software 
		\cite{ieee610.12:1990} define al requisito como:
		\begin{enumerate}[(1)]
			\item Una condici�n o capacidad que necesita un usuario para resolver un problema o 
			alcanzar un objetivo.
			\item Una condici�n o capacidad que deber�a ser reunida o pose�da por un sistema o 
			componente de un sistema para satisfacer un contrato, est�ndar, u otro documento impuesto
			formalmente.
			\item Una representaci�n documentada de una condici�n o capacidad como las 
			expresadas en (1) y (2).
		\end{enumerate}
		\item{\textbf{Tarea:}} Una secuencia de instrucciones tratadas como una unidad b�sica de 
		trabajo 
		\item{\textbf{HelpDesk:}} O en espa�ol Mesa de ayuda establece un punto �nico de contacto
		 y permite dar soporte remoto a los usuarios mejorando su productividad. Dentro de los
		 subservicios que puede brindar la mesa de servicios TI est�n: atenci�n de llamadas, soporte
		 con control remoto, gesti�n de activos, distribuci�n remota de software, soporte a sistemas de
		 antivirus, aplicaciones de autoservicio: cat�logo electr�nico y reset de passwords y respaldo 	
		 online \cite{helpdesk}.
		\item{\textbf{SCM:}}
	\end{itemize}

\section{Est�ndares para la descripci�n de Requerimientos}

\section{Herramientas para la Solicitudes de Requerimientos}
En este trabajo se entiende por Herramienta para la Solicitud de Requerimientos, 
como una plataforma a trav�s de la cual los usuarios de los sistemas de la Universidad 
de Valpara�so (Portal de Alumnos, Portal de Profesores, SCA, el Aula Virtual, SharePoint, 
etc), pueden enviar solicitudes de requerimientos (Ej: de informaci�n, soluci�n de 
problemas, cambios de clave), las cuales deben ser contestadas y resueltas 
por DISICO a la brevedad, y que a su vez permite a los Jefes de �rea asignar 
responsables a las solicitudes y mantener un visi�n clara de cuantas y cuales 
solicitudes tiene asignada cada miembro de su �rea.
\\\\
En este �mbito el tipo de herramientas en el mercado, que mas se adecua a 
este prop�sito son los Sistemas de Mesa de Ayuda (Help Desk System) o de 
Asignamiento de tickets. Ambos se centran principalmente en el seguimiento 
de problemas o de solicitudes de asistencia mediante el creaci�n y asignaci�n
de Tickets. A continuaci�n se nombran y describen algunas de las mas relevantes
de este tipo:

	\subsection{Hesk}
	Es un sistema gratuito, programado en php con mysql, que permite gestionar 
	los tickets enviados por los clientes para poder tener organizadas todas las 
	solicitaciones de nuevas funcionalidades o problemas detectados en nuestros 
	productos o servicios. La versi�n gratuita es completamente 
	funcional, aunque incluye algunas referencias a hesk.com. Sus principales 
	caracter�sticas son \cite{hesk}:
	\begin{itemize}
		\item F�cil administraci�n, con posibilidad de tener m�s de un responsable 
		por los tickets recibidos.
		\item Ilimitadas categor�as.
		\item Posibilidad de adjuntar archivos en los tickets.
		\item Sistema de anti-spam.
		\item Campos personalizados.
		\item Traducci�n sencilla a varios idiomas.
		\item Alertas por email.
	\end{itemize}

	\begin{figure}[H]
		\begin{center}
			\includegraphics[scale=0.5]{imagenes/hesk_interface.png}
			\caption{Interface de Administraci�n de Hesk}
		\end{center}
	\end{figure}

	\subsection{}
	
\section{Estandares para SCM}
\section{Herramientas para SCM}
\section{Metodologia actuales de DISICO}