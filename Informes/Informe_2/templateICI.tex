\documentclass[12pt,letterpaper]{report}

\marginparsep 0pt
\textwidth 6in
\topmargin 0pt
\headsep .5in
\textheight 9.2in
\voffset = 0pt
\hoffset = 0pt
\marginparwidth = 0pt \oddsidemargin = 0pt \sloppy

%Dimensiones de la p�gina
\usepackage[left=2.5cm,top=3cm,right=2.5cm,bottom=2.5cm]{geometry}
%Sangr�a
\setlength{\parindent}{1cm}

%Numeracion
\pagenumbering{arabic}

\usepackage{templateICI}
\usepackage{amsmath,amsfonts}
\usepackage{graphicx}
\usepackage{graphics}
\usepackage[dvips]{epsfig}
\usepackage{times}
\usepackage[latin1]{inputenc}
\usepackage[dvips]{graphicx}
\usepackage[usenames]{color}
\usepackage[spanish]{babel}
\newcommand{\ie}{i.e.}
\newcommand {\out}[1]{}
\newtheorem{definicion}{Definicion}
\usepackage[dvips]{epsfig}
\usepackage{rotating}
\usepackage{multirow}
\usepackage{array}
\usepackage{longtable}
\usepackage[]{fontenc}
\usepackage{hyperref}

\usepackage{enumerate}
\usepackage{url}
\usepackage{float}
\usepackage{booktabs}

\renewcommand{\shorthandsspanish}{}

\addto\captionsspanish{
\def\listtablename{�ndice de tablas}
\def\tablename{Tabla}}

\sloppy

\begin{document}
\title{\textbf{DESARROLLO DE UNA PLATAFORMA PARA LA SOLICITUD Y GESTI�N DE REQUERIMIENTOS Y SCM}}
\author{\textbf{Alejandro Alvarez Ahumada}}
\principaladviser{Carlos Becerra Castro}
\coprincipaladviser{Nombre Profesor Correferente}
\firstreader{Nombre Profesor Informante 1}

\beforepreface
\prefacesection{Resumen}
La Direcci�n de Servicios de Informaci�n y Computaci�n (DISICO)
de la Universidad de Valpara�so durante los �ltimos a�os ha estado 
en constante crecimiento y en busca de mejoras que le permitan 
brindar un mejor servicio. Aunque en este poco tiempo son muchas 
las mejoras que se han hecho, a�n quedan aspectos por mejorar, algunos 
de estos son los procesos relacionados a las solicitudes de requerimientos y 
solicitudes de cambios, para las cuales ya se han dise�ado procedimientos y 
metodolog�as, sin embargo se carece de una herramienta que permita la 
automatizaci�n de estas. El prop�sito de este trabajo de t�tulo es dar soluci�n 
a dicho problema mediante el desarrollo de una plataforma que permita 
automatizar los procedimientos actuales. Los principales resultados que se 
esperan son disminuir el tiempo y esfuerzo invertido en la aplicaci�n de las metodolog�as que existen actualmente.


\newpage
\prefacesection{Agradecimientos}
Aqu� pueden colocar sus agradecimientos.
Si han estudiado con becas es recomendable colocar los agradecimientos a las instituciones que les otorgaron las becas.

\afterpreface

%Aqui deben incluir el fuente de cada capitulo, sin su encabezado.
\chapter{Introducci\'on}

\label{intr}
La Direcci�n de Servicios de Informaci�n y Computaci�n (DISICO) \cite{disico}, con el objetivo de 
dar una mejor calidad de servicio, actualmente esta dividida en 3 �reas: �rea de Sistemas Financiero-Contables, 
�rea de Desarrollo y �rea de Redes, Comunicaciones y Soporte. Las funciones de las que DISICO 
es responsable se encuentran descritas en detalle en el Decreto 427 \cite{decreto427}, siendo algunas de
estas:
	\begin{itemize}
		\item Administrar todo el procesamiento de datos y documentaci�n, que por medio de sistemas 
		computacionales requiera la Universidad para su toma de decisiones.
		\item Establecer un catastro renovable en el tiempo de los requerimientos 
		inform�ticos de los usuarios de las distintas unidades de la Universidad.
		\item Interrelacionar los sistemas con las otras �reas de desarrollo de la 
		organizaci�n. 
		\item Mantener en constante actualizaci�n los sistemas de informaci�n y propender 
		a la creaci�n y desarrollo de sistemas en ambientes corporativos.
		\item Establecer pautas para obtener una estandarizaci�n en los sistemas computacionales.
	\end{itemize}

DISICO se encuentra en un proceso constante de crecimiento y mejora, para dar un mejor servicio a toda
la comunidad de la Universidad de Valpara�so. En esta b�squeda constante de aspectos a mejorar, es que 
se han realizado mejoras como por ejemplo, el Desarrollo de Metodolog�as para Software Quality Assurance (SQA)  
y Software Configuration Management (SCM) \cite{paloma}.
Pero a pesar de esto, se han detectado falencias tanto en los procesos de solicitudes de requerimientos, 
las cuales se realizan principalmente a trav�s del correo institucional y en las solicitudes de cambios, 
las que cuentan con la metodolog�a antes mencionada, pero aun no cuentan con una herramienta 
que permita su automatizaci�n.
\\\\
Por tanto se plantea dar soluci�n a dichas falencias y los problemas que estas producen
a trav�s del desarrolla de una plataforma que le permita automatizar los procesos existentes,
la cual permitir� un mejor control tanto del ciclo de vida de las tareas que se desarrollan en DISICO,
como de quienes las realizan.
 \\\\
Este documento se estructurara de la siguiente forma: 
	
\chapter{Marco Conceptual}

\section{Conceptos y Terminologia}
	\begin{itemize}
		\item{\textbf{Solicitud:}}
		\item{\textbf{Requerimiento:}} El Glosario de Terminolog�a Est�ndar de Ingenier�a de Software 
		\cite{ieee610.12:1990} define al requisito como:
		\begin{enumerate}[(1)]
			\item Una condici�n o capacidad que necesita un usuario para resolver un problema o 
			alcanzar un objetivo.
			\item Una condici�n o capacidad que deber�a ser reunida o pose�da por un sistema o 
			componente de un sistema para satisfacer un contrato, est�ndar, u otro documento impuesto
			formalmente.
			\item Una representaci�n documentada de una condici�n o capacidad como las 
			expresadas en (1) y (2).
		\end{enumerate}
		\item{\textbf{Tarea:}} Una secuencia de instrucciones tratadas como una unidad b�sica de 
		trabajo 
		\item{\textbf{HelpDesk:}} O en espa�ol Mesa de ayuda establece un punto �nico de contacto
		 y permite dar soporte remoto a los usuarios mejorando su productividad. Dentro de los
		 subservicios que puede brindar la mesa de servicios TI est�n: atenci�n de llamadas, soporte
		 con control remoto, gesti�n de activos, distribuci�n remota de software, soporte a sistemas de
		 antivirus, aplicaciones de autoservicio: cat�logo electr�nico y reset de passwords y respaldo 	
		 online \cite{helpdesk}.
		\item{\textbf{SCM:}}
	\end{itemize}

\section{Est�ndares para la descripci�n de Requerimientos}

\section{Herramientas para la Solicitudes de Requerimientos}
En este trabajo se entiende por Herramienta para la Solicitud de Requerimientos, 
como una plataforma a trav�s de la cual los usuarios de los sistemas de la Universidad 
de Valpara�so (Portal de Alumnos, Portal de Profesores, SCA, el Aula Virtual, SharePoint, 
etc), pueden enviar solicitudes de requerimientos (Ej: de informaci�n, soluci�n de 
problemas, cambios de clave), las cuales deben ser contestadas y resueltas 
por DISICO a la brevedad, y que a su vez permite a los Jefes de �rea asignar 
responsables a las solicitudes y mantener un visi�n clara de cuantas y cuales 
solicitudes tiene asignada cada miembro de su �rea.
\\\\
En este �mbito el tipo de herramientas en el mercado, que mas se adecua a 
este prop�sito son los Sistemas de Mesa de Ayuda (Help Desk System) o de 
Asignamiento de tickets. Ambos se centran principalmente en el seguimiento 
de problemas o de solicitudes de asistencia mediante el creaci�n y asignaci�n
de Tickets. A continuaci�n se nombran y describen algunas de las mas relevantes
de este tipo:

	\subsection{Hesk}
	Es un sistema gratuito, programado en php con mysql, que permite gestionar 
	los tickets enviados por los clientes para poder tener organizadas todas las 
	solicitaciones de nuevas funcionalidades o problemas detectados en nuestros 
	productos o servicios. La versi�n gratuita es completamente 
	funcional, aunque incluye algunas referencias a hesk.com. Sus principales 
	caracter�sticas son \cite{hesk}:
	\begin{itemize}
		\item F�cil administraci�n, con posibilidad de tener m�s de un responsable 
		por los tickets recibidos.
		\item Ilimitadas categor�as.
		\item Posibilidad de adjuntar archivos en los tickets.
		\item Sistema de anti-spam.
		\item Campos personalizados.
		\item Traducci�n sencilla a varios idiomas.
		\item Alertas por email.
	\end{itemize}

	\begin{figure}[H]
		\begin{center}
			\includegraphics[scale=0.5]{imagenes/hesk_interface.png}
			\caption{Interface de Administraci�n de Hesk}
		\end{center}
	\end{figure}

	\subsection{osTicket}
	Es un sistema de ticket open source muy extendido. Integra sin problemas las preguntas creadas  
	via email, tel�fono y a traves de formularios web en una interfaz simple y f�cil de usar. Gestiona, 
	organiza y archiva todas las solicitudes de apoyo en un solo lugar, mientras que provee a los clientes 
	de la capacidad de respuesta que merecen. osTicket est� dise�ado para ayudar a agilizar 
	las solicitudes de apoyo y mejorar la eficiencia de atenci�n al cliente, proporcionando al personal 
	las herramientas necesarias para ofrecer un soporte t�cnico r�pido, eficaz y medible. 
	Algunas de las caracter�sticas principales incluyen:
	\begin{itemize}
		\item Los tickets pueden ser creados a trav�s de emails, formularios online o 
		por tel�fono (creado por el staff).
		\item Respuesta autom�tica que se env�a cuando un nuevo ticket es abierto 
		o un mensaje es recibido (plantillas personalizables de correo).
		\item Respuestas predefinidas para las preguntas mas frecuentes.
		\item A�adir notas internas a los tickets para el personal.
		\item Temas de ayuda configurables para los tickets web. Enruta las 
		consultas sin exponer los departamentos internos y prioridades.
		\item El personal y los clientes se mantienen al dia con alertas de correo 
		electr�nico (ajuste configurable y felxible).
		\item Controla los niveles de acceso del personal, basado en grupos y 
		departamentos.
		\item Asigna tickets al personal y/o departamentos.
		\item No requiere cuentas de usuarios o registro de usuarios 
		(Ticket ID/email usados para el login).
		\item Todas las solicitudes de apoyo y respuesta son archivadas.
	\end{itemize}
	
	\begin{figure}[H]
		\begin{center}
			\includegraphics[scale=1]{imagenes/osTicket_interface.png}
			\caption{Interface de Vision y Respuesta de Tickets en osTicket}
		\end{center}
	\end{figure}

	\subsection{OTRS}
	OTRS (Open-source Ticket Request System) es una suite de c�digo abierto 
	l�der en innovaci�n de servicios, que incluye Help Desk, una soluci�n para la
	gesti�n de servicios de IT. Es conocido por su administraci�n de solicitudes 
	de servicio. El Panel Principal de OTRS proporciona un completo equipo de
	herramientas para visualizar, clasificar, administrar, responder, escalar y 
	resolver las solicitudes de servicio. Cuenta con una amplia gama de funciones
	disponibles y se integra sin problemas con procesos existentes. Algunas de las 
	principales funciones con las que cuenta son:
	\begin{itemize}
		\item Generaci�n de Tickets, con soporte de m�ltiples entradas (correo, fax, 
		PDA, SMS o SOAP/XML).
		\item Clasificaci�n y priorizaci�n de Tickets configurables.
		\item Interfaz de auto servicio del cliente.
		\item Notificaciones por evento de cambios en el estado de los tickets.
		\item Plantillas de auto respuesta, para solicitudes recurrentes. 		
		\item Enrutamiento  autom�tico y manual de incidencias.
		\item Integraci�n del centro de asistencia telef�nica.
		\item Ejecuci�n autom�tica de acciones pre definidas 
		utilizando filtros de criterio.
		\item M�ltiples Visualizaciones.
		\item B�squeda de texto completo por �ndice en todos los tickets: 
		por t�tulo, hora, cliente o agente.
	\end{itemize}
	
	\begin{figure}[H]
		\begin{center}
			\includegraphics[scale=0.4]{imagenes/OTRS_interface.png}
			\caption{Interface de Vista de Tickets en OTRS}
		\end{center}
	\end{figure}
	\subsection{}
\section{Estandares para SCM}
\section{Herramientas para SCM}
\section{Metodologia actuales de DISICO}
\chapter{Definici�n del Problema}


\section{Situaci�n Actual}

	\subsection{Metodolog�a Actual para la Solicitud de Requerimientos}

	\subsection{Metodolog�a Actual Gesti�n de Requerimientos en los Proyectos}
	
	\subsection{Metodolog�a Actual de SCM}
	La actual metodolog�a de SCM utilizada en DISICO, fue desarrollada y descrita
	en el Trabajo de Titulo \textit{Desarrollo de Metodolog�as de SQA y SCM para 
	la Direcci�n de Servicios de Informaci�n y Computaci�n}\cite{paloma}. Las 
	principales etapas que componen esta metodolog�a se encuentran representadas
	en la Tabla \ref{scm_paloma}
	
	\begin{table}[H]
		\begin{tabular}{| l | l |}
			\hline
			\textbf{Etapa} & \textbf{Descripci�n}\\
			\hline
			\multirow{2}{*}{Tareas y Responsables} & {Identificar tareas asociadas y asignar responsables}\\
									                       & {a cada una.}\\
			\hline
			{Actividades SCM} & {Se definen las actividades principales a realizar.}\\
			\hline
			{Recursos} & {Herramientas de apoyo a la metodolog�a}\\
			\hline
			\multirow{2}{*}{Formaci�n} & {Material visual para educar a los integrantes de cada proyecto}\\
                         							& {en cuanto a SCM.}\\
			\hline
		\end{tabular}
		\caption{Etapas de la actual metodolog�a de SCM}
		\label{scm_paloma}
	\end{table}

	
\section{Formulaci�n del Problema}

La Direcci�n de Servicios de Informaci�n y Computaci�n (DISICO) de la 
Universidad de Valpara�so en su constante b�squeda por mejorar ha detectado 
algunos problemas, los cuales se nombran y detallan a continuaci�n:
	
	\subsection{Problemas en las Solicitudes de Requerimientos}
	\label{prob1}
	Las solicitudes de requerimientos que se reciben en DISICO, actualmente deben 
	ser enviadas a trav�s del correo electr�nico institucional al Jefe de �rea, quien revisa y eval�a 
	cada solicitud para determinar quien realizara la tarea correspondiente, luego reenv�a el 
	correo original a dicha persona, en algunos casos adjuntando informaci�n adicional 
	sobre como completar la tarea o en caso que la solicitud enviada en el correo este
	incompleta como para poder comprender o resolver completamente el requerimiento, 
	se reenv�a al funcionario el correo que ha enviado junto con las observaciones correspondientes
	\\\\
	Aunque a primera vista parece simple, esto conlleva los siguientes problemas:
	\begin{itemize}
		\item El correo del Jefe de �rea puede verse saturado con cientos de correos 
		en un mismo d�a. Sobretodo en periodos cr�ticos como son el inicio y termino de 
		cada semestre.
		\item Los requerimientos quedan almacenados en emails y no se cuenta con 
		ninguna estructura de datos que permita, mejores b�squedas, priorizaci�n de 
		tareas, an�lisis de avance de las tareas, an�lisis posteriores acerca de velocidad 
		de desarrollo o productividad, establecer dependencias entre tareas, 
		complejidad de las tareas, etc.
		\item El Jefe de �rea no tiene una visi�n clara de cuales ni cuantas tareas se 
		han asignado a cada miembro del equipo.
		\item El Jefe de �rea no tiene una visi�n clara sobre cuales ni cuantas tareas 
		est�n a�n en espera, en desarrollo o finalizadas.
		\item Los funcionarios que env�an solicitudes de requerimientos desconocen en 
		que estado se encuentran, cuanto se ha avanzado, ni la prioridad que esta 
		tiene, y menos las complicaciones que pueden haber surgido, por lo que deben 
		estar consultando por tel�fono o esperando asta recibir un respuesta.
		\item Entre tanto correo muchas veces se pierden los requerimientos debido a 
		que el responsable ley� el correo y dejo la tarea para mas tarde debido a que 
		en ese momento a�n se encontraba realizando otras tareas.
	\end{itemize}

	\subsection{Problemas en la Gesti�n de Requerimientos en los \\Proyectos}
	\label{prob2}
	Adem�s de los requerimientos que se reciben diariamente en DISICO, existen 
	requerimientos que superan esta categor�a, los cuales pasan a convertirse en 
	proyectos, estos nuevos proyectos constan de un conjunto 	de requerimientos 
	que son capturados por los responsables de dicho proyecto, quienes los agrupan y 
	calendarizan mediante una Carta Gantt, la cual debe ser entregada al Jefe de �rea.
	\\\\
	Este sistema tambi�n presenta algunas falencias, las cuales son similares a las 
	presentadas en el punto anterior y se detallan a continuaci�n:
	
	\begin{itemize}
		\item Aunque la Carta Gantt es un instrumento importante, no le permite al 
		Jefe de �rea conocer cuales puntos ya han sido completados, en cuales se esta 
		trabajado actualmente y cual es el nivel de avance de estos.
		\item Tampoco tiene un registro de las causas que pueden estar demorando 
		una tarea.
		\item Esto tambi�n implica que el Jefe de �rea no tiene una visi�n clara de la 
		productividad ni del avance de cada uno de los involucrados en 
		un proyecto.
	\end{itemize}

	\subsection{Problemas en la Solicitudes de SCM}
	\label{prob3}
	Antiguamente DISICO no contaba con un procedimiento formal para los 
	procesos de SCM, aunque se intentaba llevar a cabo un procedimiento similar al 
	de asignaci�n de tareas, pero con algunas peque�as diferencias. Los problemas 
	de este procedimiento fueron analizados y corregidos en el Trabajo de Titulo 
	Desarrollo de Metodolog�as de SQA y SCM para la Direcci�n de Servicios de Informaci�n 
	y Computaci�n \cite{paloma}, en el cual se propuso una nueva metodolog�a de SCM, 
	en la cual ya fueron considerados la baja disponibilidad de tiempo del personal para 
	aplicar estos procesos y la falta de recursos para las mismas.
	\\\\
	En dicho trabajo para lograr un proceso de SCM de calidad se creo toda 
	la documentaci�n pertinente a los cambios que se realicen durante el desarrollo 
	de un proyecto, pero siempre buscando que �sta se realice de forma r�pida y 
	precisa, apoyada de templates que exijan la informaci�n m�nima para que la 
	metodolog�a cumpla sus objetivos.
	\\\\
	Sin embargo aun con esta nueva metodolog�a se tienen algunos problemas 
	los cuales se detallan a continuaci�n:
	
	\begin{itemize}
		\item La metodolog�a propuesta no logra ser lo suficientemente �gil, debido a 
		que creaci�n de la documentaci�n que exige, no se encuentra automatizada.
		\item La documentaci�n se realiza en plantillas Word, por lo que la persona deber� descargar 
		la plantilla si es que no la posee, abrirla con Word (o similar) para editarla, luego imprimirla y archivarla 
		o si se desea compartirlas mantenerlas versionadas en SVN, de donde el resto de los involucrados en los 
		cambios deber�n estar descargando la ultima versi�n para conocer los nuevos cambios, y 
		actualizarlas si es necesario.
		\item El uso de estas plantillas Word no facilita la b�squeda r�pida de 
		informaci�n sobre cambios pasados, personas involucradas en los cambios, etc. 
		Debido a que se debe descargar los �ltimos cambios de los documentos, buscar por los nombres de los documentos, 
		cuales posiblemente est�n relacionados a la b�squeda en particular y abrirlos para 
		buscar dentro de estos lo que se necesita.
	\end{itemize} 
	
	Todas estas problem�ticas han provocado que no se est� sacando el m�ximo provecho a 
	la metodolog�a dise�ada y como podemos ver la causa principal es que no 
	se posee una herramienta adecuada que le de soporte.
\section{Soluci�n Propuesta}

\section{Objetivos}

	\subsection{Objetivo General}
	El objetivo es el desarrollo de una aplicaci�n para la solicitud y gesti�n de requerimientos y SCM para 
	la Direcci�n de Servicios de Informaci�n y Computaci�n de la Universidad de Valpara�so (DISICO).

	\subsection{Objetivos Espec�ficos}
	\begin{itemize}
		\item Dar soporte mediante una aplicaci�n web, a la Metodolog�a de SCM 
		dise�ada para DISICO \cite{paloma}.
		\item Permitir monitorear el ciclo de vida completo de cada solicitud de 
		requerimiento y de cambio que se recibe en DISICO.
		\item Mejorar el control que tiene el Jefe de �rea sobre las labores que se 
		encuentra desempe�ando actualmente cada miembro del equipo.
		\item Mantener un historial de todas los requerimientos que se han 
		completado, para permitir mejorar las estimaciones de tiempo y esfuerzo, para 
		futuros requerimientos de similares caracter�sticas.
		\item Obtener mediciones cuantitativas de la productividad del equipo de desarrollo.  
	\end{itemize}
\chapter{An�lisis}

\section{Especificaci�n de Requerimientos}
	A continuaci�n se describen los requerimientos funcionales (RF) y tambi�n
	los requerimientos no funcionales (RNF), los cuales describen las funciones
	y caracter�sticas que debe poseer la plataforma que se debe desarrollar.
	
	\subsection{Definiciones, Acronimos y Abreviaturas}
	
	\subsection{Tipos de Usuarios}
	Dentro del sistema a desarrollar se pueden encontrar diferentes tipos de usuarios,
	los cuales son definidos y descritos a continuaci�n:
	
	\begin{itemize}
		\item \textbf{Administrador:} Usuario con conocimientos avanzados en 
		computaci�n. Encargado de administrar el sistema, con acceso a todas las 
		funcionalidades del sistema.
		\item \textbf{Jefe de Departamento:} Usuario con conocimientos generales
		de computaci�n. Quien es el jefe de todo DISICO.
		\item \textbf{Jefe de �rea:} Jefe de alguno de los departamentos de DISICO, 
		encargado de asignar responsables a las solicitudes y de controlar que estas 
		se lleven a cabo.
		\item \textbf{Funcionario de DISICO:} Usuarios con conocimientos avanzados
		en computaci�n, y que deben resolver las solicitudes que les son
		asignadas. Estos tambi�n pueden tener asignada alguna de las tareas de SCM,
		definidas por la metodolog�a.
		\item \textbf{Solicitante:} Usuario con conocimientos b�sicos en computaci�n,
		quienes env�an a DISICO solicitudes de requerimientos.
	\end{itemize}

	\subsection{Requerimientos Funcionales}
	\begin{center}
	\begin{tabular}{| l | m{1.5cm} | l | m{4cm} | l | m{2.2cm} |}
		\hline
		\textbf{Id} & {RF01} & \textbf{Tipo de Usuario} & {Todos} & \textbf{Prioridad} & {Obligatorio}\\
		\hline
		\multicolumn{2}{|l|}{\textbf{Descripci�n}} & \multicolumn{4}{m{12cm}|}{Autenticar a los usuarios 
		con sus cuentas institucionales.}\\
		\hline
		\multicolumn{2}{|l|}{\textbf{Entrada}} & \multicolumn{4}{m{12cm}|}{Rut y Contrase�a.}\\
		\hline
		\multicolumn{2}{|l|}{\textbf{Proceso}} & \multicolumn{4}{m{12cm}|}{Se verifica que el rut y la contrase�a 
		concuerden y si tiene los permisos necesarios.}\\
		\hline
		\multicolumn{2}{|l|}{\textbf{Salida}} & \multicolumn{4}{m{12cm}|}{Ingreso al sistema con el perfil ingresado.}\\
		\hline
	\end{tabular}	 
	\end{center}

	\begin{center}
	\begin{tabular}{| l | m{1.5cm} | l | m{4cm} | l | m{2.2cm} |}
		\hline
		\textbf{Id} & {RF02} & \textbf{Tipo de Usuario} & {Solicitante} & \textbf{Prioridad} & {Obligatorio}\\
		\hline
		\multicolumn{2}{|l|}{\textbf{Descripci�n}} & \multicolumn{4}{m{12cm}|}{Crear y enviar solicitudes de 
		requerimientos, dirigida a cualquiera de las 3 �reas de DISICO (Desarrollo, Fincom o Redes).}\\
		\hline
		\multicolumn{2}{|l|}{\textbf{Entrada}} & \multicolumn{4}{m{12cm}|}{Datos de la solicitud (Asunto, 
		Descripci�n de la solicitud, tipo de solicitud, departamento de DISICO a la que esta dirigida, sistema 
		al que se refiere).}\\
		\hline
		\multicolumn{2}{|l|}{\textbf{Proceso}} & \multicolumn{4}{m{12cm}|}{Se registra y almacena la solicitud, 
		y se genera.}\\
		\hline
		\multicolumn{2}{|l|}{\textbf{Salida}} & \multicolumn{4}{m{12cm}|}{Pantalla de envi� exitoso y 
		notificaci�n en el perfil del Jefe de �rea correspondiente.}\\
		\hline
	\end{tabular}
	\end{center}
	
	\begin{center}
	\begin{tabular}{| l | m{1.5cm} | l | m{4cm} | l | m{2.2cm} |}
		\hline
		\textbf{Id} & {RF03} & \textbf{Tipo de Usuario} & {Solicitante} & \textbf{Prioridad} & {Obligatorio}\\
		\hline
		\multicolumn{2}{|l|}{\textbf{Descripci�n}} & \multicolumn{4}{m{12cm}|}{Enviar al email del solicitante 
		el numero de consulta de su solicitud.}\\
		\hline
		\multicolumn{2}{|l|}{\textbf{Entrada}} & \multicolumn{4}{m{12cm}|}{Registro de una solicitud.}\\
		\hline
		\multicolumn{2}{|l|}{\textbf{Proceso}} & \multicolumn{4}{m{12cm}|}{Al registrar la solicitud, autom�ticamente 
		se genera un numero de consulta para esta, el cual es enviado autom�ticamente al usuario.}\\
		\hline
		\multicolumn{2}{|l|}{\textbf{Salida}} & \multicolumn{4}{m{12cm}|}{Correo electr�nico con el numero 
		de consulta de la solicitud.}\\
		\hline
	\end{tabular}
	\end{center}

	\begin{center}
	\begin{tabular}{| l | m{1.5cm} | l | m{4cm} | l | m{2.2cm} |}
		\hline
		\textbf{Id} & {RF04} & \textbf{Tipo de Usuario} & {Solicitante} & \textbf{Prioridad} & {Obligatorio}\\
		\hline
		\multicolumn{2}{|l|}{\textbf{Descripci�n}} & \multicolumn{4}{m{12cm}|}{Buscar solicitud en el historial}\\
		\hline
		\multicolumn{2}{|l|}{\textbf{Entrada}} & \multicolumn{4}{m{12cm}|}{Par�metros de b�squeda 
		(Numero, Fecha o Asunto)}\\
		\hline
		\multicolumn{2}{|l|}{\textbf{Proceso}} & \multicolumn{4}{m{12cm}|}{Se busca entre todos las 
		solicitudes generadas por dicho solicitante aquellas que coincidan con los par�metros de b�squeda.}\\
		\hline
		\multicolumn{2}{|l|}{\textbf{Salida}} & \multicolumn{4}{m{12cm}|}{Listado con todas las solicitudes resultantes.}\\
		\hline
	\end{tabular}
	\end{center}

	\begin{center}
	\begin{tabular}{| l | m{1.5cm} | l | m{4cm} | l | m{2.2cm} |}
		\hline
		\textbf{Id} & {RF05} & \textbf{Tipo de Usuario} & {Jefe de �rea} & \textbf{Prioridad} & {Obligatorio}\\
		\hline
		\multicolumn{2}{|l|}{\textbf{Descripci�n}} & \multicolumn{4}{m{12cm}|}{Asignar un responsable 
		para una solicitud recibida.}\\
		\hline
		\multicolumn{2}{|l|}{\textbf{Entrada}} & \multicolumn{4}{m{12cm}|}{Responsable, prioridad y 
		fecha de vencimiento.}\\
		\hline
		\multicolumn{2}{|l|}{\textbf{Proceso}} & \multicolumn{4}{m{12cm}|}{Se registran la informaci�n 
		ingresada y se notifica al responsable}\\
		\hline
		\multicolumn{2}{|l|}{\textbf{Salida}} & \multicolumn{4}{m{12cm}|}{Notificaci�n en el perfil del responsable.}\\
		\hline
	\end{tabular}
	\end{center}

	\begin{center}
	\begin{tabular}{| l | m{1.5cm} | l | m{4cm} | l | m{2.2cm} |}
		\hline
		\textbf{Id} & {RF06} & \textbf{Tipo de Usuario} & {Jefe de �rea} & \textbf{Prioridad} & {Obligatorio}\\
		\hline
		\multicolumn{2}{|l|}{\textbf{Descripci�n}} & \multicolumn{4}{m{12cm}|}{Transferir una solicitud 
		a otra �rea en caso de que esta venga mal asignada.}\\
		\hline
		\multicolumn{2}{|l|}{\textbf{Entrada}} & \multicolumn{4}{m{12cm}|}{�rea y Justificaci�n de la Transferencia.}\\
		\hline
		\multicolumn{2}{|l|}{\textbf{Proceso}} & \multicolumn{4}{m{12cm}|}{Se registra la transferencia de la solicitud}\\
		\hline
		\multicolumn{2}{|l|}{\textbf{Salida}} & \multicolumn{4}{m{12cm}|}{Notificaci�n en el perfil del Jefe de 
		�rea donde fue transferida la solicitud.}\\
		\hline
	\end{tabular}
	\end{center}

	\begin{center}
	\begin{tabular}{| l | m{1.5cm} | l | m{4cm} | l | m{2.2cm} |}
		\hline
		\textbf{Id} & {RF07} & \textbf{Tipo de Usuario} & {Jefe de �rea} & \textbf{Prioridad} & {Obligatorio}\\
		\hline
		\multicolumn{2}{|l|}{\textbf{Descripci�n}} & \multicolumn{4}{m{12cm}|}{Rechazar una solicitud.}\\
		\hline
		\multicolumn{2}{|l|}{\textbf{Entrada}} & \multicolumn{4}{m{12cm}|}{Justificaci�n del rechazo.}\\
		\hline
		\multicolumn{2}{|l|}{\textbf{Proceso}} & \multicolumn{4}{m{12cm}|}{Se registra la justificaci�n 
		del rechazo y se cierra la solicitud.}\\
		\hline
		\multicolumn{2}{|l|}{\textbf{Salida}} & \multicolumn{4}{m{12cm}|}{Correo electr�nico al solicitante 
		notificando el rechazo de su solicitud.}\\
		\hline
	\end{tabular}
	\end{center}

	\begin{center}
	\begin{tabular}{| l | m{1.5cm} | l | m{4cm} | l | m{2.2cm} |}
		\hline
		\textbf{Id} & {RF08} & \textbf{Tipo de Usuario} & {Jefe de �rea} & \textbf{Prioridad} & {Deseable}\\
		\hline
		\multicolumn{2}{|l|}{\textbf{Descripci�n}} & \multicolumn{4}{m{12cm}|}{Convertir una solicitud en un proyecto.}\\
		\hline
		\multicolumn{2}{|l|}{\textbf{Entrada}} & \multicolumn{4}{m{12cm}|}{Nombre del Proyecto y Responsables.}\\
		\hline
		\multicolumn{2}{|l|}{\textbf{Proceso}} & \multicolumn{4}{m{12cm}|}{Se registra el proyecto y se cierra la solicitud.}\\
		\hline
		\multicolumn{2}{|l|}{\textbf{Salida}} & \multicolumn{4}{m{12cm}|}{Correo electr�nico notificando al solicitante.}\\
		\hline
	\end{tabular}
	\end{center}
	
	\begin{center}
	\begin{tabular}{| l | m{1.5cm} | l | m{4cm} | l | m{2.2cm} |}
		\hline
		\textbf{Id} & {RF09} & \textbf{Tipo de Usuario} & {Jefe de �rea} & \textbf{Prioridad} & {Obligatorio}\\
		\hline
		\multicolumn{2}{|l|}{\textbf{Descripci�n}} & \multicolumn{4}{m{12cm}|}{Ver resumen de las 
		solicitudes del �rea, el cual debe mostrar las cantidad de solicitudes no asignadas, asignadas, iniciadas, 
		pendientes, atrasadas, finalizadas, rechazadas y transferidas del �rea y por responsable a la fecha.}\\
		\hline
		\multicolumn{2}{|l|}{\textbf{Entrada}} & \multicolumn{4}{m{12cm}|}{Se selecciona la opci�n.}\\
		\hline
		\multicolumn{2}{|l|}{\textbf{Proceso}} & \multicolumn{4}{m{12cm}|}{Se calculan la cantidad 
		de solicitudes en los diferentes estados del �rea y por responsable.}\\
		\hline
		\multicolumn{2}{|l|}{\textbf{Salida}} & \multicolumn{4}{m{12cm}|}{Se despliega la informaci�n 
		en pantalla en forma de tabla y gr�fico.}\\
		\hline
	\end{tabular}
	\end{center}
	
	\begin{center}
	\begin{tabular}{| l | m{1.5cm} | l | m{4cm} | l | m{2.2cm} |}
		\hline
		\textbf{Id} & {RF10} & \textbf{Tipo de Usuario} & {Jefe de Departamento} & \textbf{Prioridad} & {Obligatorio}\\
		\hline
		\multicolumn{2}{|l|}{\textbf{Descripci�n}} & \multicolumn{4}{m{12cm}|}{Ver resumen de las 
		solicitudes del departamento, el cual debe mostrar las cantidad de solicitudes no asignadas, asignadas, iniciadas, 
		pendientes, atrasadas, finalizadas, rechazadas y transferidas del departamento, por �rea y por responsable a la fecha .}\\
		\hline
		\multicolumn{2}{|l|}{\textbf{Entrada}} & \multicolumn{4}{m{12cm}|}{Se selecciona la opci�n.}\\
		\hline
		\multicolumn{2}{|l|}{\textbf{Proceso}} & \multicolumn{4}{m{12cm}|}{Se calculan la cantidad de 
		solicitudes en los diferentes estados del departamento, por �rea y por responsable.}\\
		\hline
		\multicolumn{2}{|l|}{\textbf{Salida}} & \multicolumn{4}{m{12cm}|}{Se despliega la informaci�n 
		en pantalla en forma de tabla y gr�fico.}\\
		\hline
	\end{tabular}
	\end{center}
	
	\begin{center}
	\begin{tabular}{| l | m{1.5cm} | l | m{4cm} | l | m{2.2cm} |}
		\hline
		\textbf{Id} & {RF11} & \textbf{Tipo de Usuario} & {Jefe de Departamento} & \textbf{Prioridad} & {Obligatorio}\\
		\hline
		\multicolumn{2}{|l|}{\textbf{Descripci�n}} & \multicolumn{4}{m{12cm}|}{Buscar solicitudes por numero de consulta 
		o por alguno de los siguientes filtros de b�squeda (�rea, Responsable, Solicitante, Estado, Fecha, Asunto, Tipo, Sistema)}\\
		\hline
		\multicolumn{2}{|l|}{\textbf{Entrada}} & \multicolumn{4}{m{12cm}|}{Par�metro de b�squeda.}\\
		\hline
		\multicolumn{2}{|l|}{\textbf{Proceso}} & \multicolumn{4}{m{12cm}|}{Se busca entre los registros aquellas solicitudes 
		que coincidan con los par�metros de b�squeda especificados.}\\
		\hline
		\multicolumn{2}{|l|}{\textbf{Salida}} & \multicolumn{4}{m{12cm}|}{Listado de las solicitudes.}\\
		\hline
	\end{tabular}
	\end{center}
	
	\begin{center}
	\begin{tabular}{| l | m{1.5cm} | l | m{4cm} | l | m{2.2cm} |}
		\hline
		\textbf{Id} & {RF12} & \textbf{Tipo de Usuario} & {Jefe de �rea y Funcionario DISICO} & \textbf{Prioridad} & {Obligatorio}\\
		\hline
		\multicolumn{2}{|l|}{\textbf{Descripci�n}} & \multicolumn{4}{m{12cm}|}{Mostrar alertas cuando existan solicitudes atrasadas.}\\
		\hline
		\multicolumn{2}{|l|}{\textbf{Entrada}} & \multicolumn{4}{m{12cm}|}{La fecha actual.}\\
		\hline
		\multicolumn{2}{|l|}{\textbf{Proceso}} & \multicolumn{4}{m{12cm}|}{Al ingresar al perfil se comprueba cuales solicitudes han 
		excedido su fecha de vencimiento y se cambian a estado atrasadas.}\\
		\hline
		\multicolumn{2}{|l|}{\textbf{Salida}} & \multicolumn{4}{m{12cm}|}{Una alerta en el perfil del Jefe de �rea y del Responsable.}\\
		\hline
	\end{tabular}
	\end{center}
	
	\begin{center}
	\begin{tabular}{| l | m{1.5cm} | l | m{4cm} | l | m{2.2cm} |}
		\hline
		\textbf{Id} & {RF13} & \textbf{Tipo de Usuario} & {Funcionario DISICO} & \textbf{Prioridad} & {Obligatorio}\\
		\hline
		\multicolumn{2}{|l|}{\textbf{Descripci�n}} & \multicolumn{4}{m{12cm}|}{El responsable podr� modificar el estado de una 
		solicitud a pendiente, iniciada o finalizada .}\\
		\hline
		\multicolumn{2}{|l|}{\textbf{Entrada}} & \multicolumn{4}{m{12cm}|}{El nuevo estado de la tarea.}\\
		\hline
		\multicolumn{2}{|l|}{\textbf{Proceso}} & \multicolumn{4}{m{12cm}|}{Se actualiza el estado de la tarea.}\\
		\hline
		\multicolumn{2}{|l|}{\textbf{Salida}} & \multicolumn{4}{m{12cm}|}{Pantalla de selecci�n de tipo de respuesta.}\\
		\hline
	\end{tabular}
	\end{center}
	
	\begin{center}
	\begin{tabular}{| l | m{1.5cm} | l | m{4cm} | l | m{2.2cm} |}
		\hline
		\textbf{Id} & {RF14} & \textbf{Tipo de Usuario} & {Funcionario DISICO} & \textbf{Prioridad} & {Obligatorio}\\
		\hline
		\multicolumn{2}{|l|}{\textbf{Descripci�n}} & \multicolumn{4}{m{12cm}|}{Al finalizar una solicitud el 
		responsable puede escoger entre 2 tipos de respuesta que son: Responder al Jefe de �rea o 
		Responder al solicitante.}	\\
		\hline
		\multicolumn{2}{|l|}{\textbf{Entrada}} & \multicolumn{4}{m{12cm}|}{Selecci�n de una opci�n de respuesta.}\\
		\hline
		\multicolumn{2}{|l|}{\textbf{Proceso}} & \multicolumn{4}{m{12cm}|}{Seg�n la opci�n se realiza alguna de las 
		siguientes acciones:
		\begin{itemize}
			\item \textbf{Responder al Jefe de �rea:} Esta opci�n solo envi� una notificaci�n al jefe de �rea quien 
			deber� encargarse de responder luego al solicitante.
			\item \textbf{Responder al solicitante:} Esta envi� un correo electr�nico al solicitante, notific�ndole 
			que su solicitud ha sido completada.
		\end{itemize}
		}\\
		\hline
		\multicolumn{2}{|l|}{\textbf{Salida}} & \multicolumn{4}{m{12cm}|}{Notificaci�n a quien corresponde seg�n el caso.}\\
		\hline
	\end{tabular}
	\end{center}
	
	\begin{center}
	\begin{tabular}{| l | m{1.5cm} | l | m{4cm} | l | m{2.2cm} |}
		\hline
		\textbf{Id} & {RF15} & \textbf{Tipo de Usuario} & {Jefe de �rea} & \textbf{Prioridad} & {Obligatorio}\\
		\hline
		\multicolumn{2}{|l|}{\textbf{Descripci�n}} & \multicolumn{4}{m{12cm}|}{Las solicitudes que deben ser
		respuestas por el, tienen la posibilidad de responderse de 2 formas, las cuales son: Respuesta Directa o
		Respuesta manual.}\\
		\hline
		\multicolumn{2}{|l|}{\textbf{Entrada}} & \multicolumn{4}{m{12cm}|}{Selecci�n de una opci�n de respuesta.}\\
		\hline
		\multicolumn{2}{|l|}{\textbf{Proceso}} & \multicolumn{4}{m{12cm}|}{Seg�n la opci�n se realiza alguna de las 
		siguientes acciones:
		\begin{itemize}
			\item \textbf{Respuesta Directa:} Se notifica directamente al solicitante a trabes de correo electr�nico.
			\item \textbf{Responder Manual:} Se envi� un correo electr�nico personalizado a una o mas personas
			entre las cuales puede estar o no el solicitante.
		\end{itemize}
		}\\
		\hline
		\multicolumn{2}{|l|}{\textbf{Salida}} & \multicolumn{4}{m{12cm}|}{Notificaci�n a quien corresponde seg�n el caso..}\\
		\hline
	\end{tabular}
	\end{center}

	\begin{center}
	\begin{tabular}{| l | m{1.5cm} | l | m{4cm} | l | m{2.2cm} |}
		\hline
		\textbf{Id} & {RF16} & \textbf{Tipo de Usuario} & {Todos} & \textbf{Prioridad} & {Obligatorio}\\
		\hline
		\multicolumn{2}{|l|}{\textbf{Descripci�n}} & \multicolumn{4}{m{12cm}|}{El responsable, 
		el solicitante y el jefe de �rea pueden agregar comentarios a una solicitud.}\\
		\hline
		\multicolumn{2}{|l|}{\textbf{Entrada}} & \multicolumn{4}{m{12cm}|}{Comentario.}\\
		\hline
		\multicolumn{2}{|l|}{\textbf{Proceso}} & \multicolumn{4}{m{12cm}|}{Se registra el comentario asoci�ndolo a la solicitud.}\\
		\hline
		\multicolumn{2}{|l|}{\textbf{Salida}} & \multicolumn{4}{m{12cm}|}{Notificaci�n en el perfil del resto de los involucrados, 
		sin considerar a quien genero la solicitud.}\\
		\hline
	\end{tabular}
	\end{center}
	
	\begin{center}
	\begin{tabular}{| l | m{1.5cm} | l | m{4cm} | l | m{2.2cm} |}
		\hline
		\textbf{Id} & {RF17} & \textbf{Tipo de Usuario} & {Jefe de �rea} & \textbf{Prioridad} & {Obligatorio}\\
		\hline
		\multicolumn{2}{|l|}{\textbf{Descripci�n}} & \multicolumn{4}{m{12cm}|}{Crear nuevo proyecto.}\\
		\hline
		\multicolumn{2}{|l|}{\textbf{Entrada}} & \multicolumn{4}{m{12cm}|}{Informaci�n del proyecto e involucrados
		, seg�n lo descrito en la metodolog�a de SCM.}\\
		\hline
		\multicolumn{2}{|l|}{\textbf{Proceso}} & \multicolumn{4}{m{12cm}|}{Se registra el nuevo proyecto.}\\
		\hline
		\multicolumn{2}{|l|}{\textbf{Salida}} & \multicolumn{4}{m{12cm}|}{Pantalla principal de configuraci�n
		del proyecto.}\\
		\hline
	\end{tabular}
	\end{center}
	
	\begin{center}
	\begin{tabular}{| l | m{1.5cm} | l | m{4cm} | l | m{2.2cm} |}
		\hline
		\textbf{Id} & {RF18} & \textbf{Tipo de Usuario} & {Jefe de �rea} & \textbf{Prioridad} & {Obligatorio}\\
		\hline
		\multicolumn{2}{|l|}{\textbf{Descripci�n}} & \multicolumn{4}{m{12cm}|}{Definir responsables para las
		tareas de SCM del proyecto, definidas por la metodolog�a.}\\
		\hline
		\multicolumn{2}{|l|}{\textbf{Entrada}} & \multicolumn{4}{m{12cm}|}{Responsable y tarea.}\\
		\hline
		\multicolumn{2}{|l|}{\textbf{Proceso}} & \multicolumn{4}{m{12cm}|}{Se registra el responsable de la tarea.}\\
		\hline
		\multicolumn{2}{|l|}{\textbf{Salida}} & \multicolumn{4}{m{12cm}|}{Pantalla de asignaci�n de tareas de SCM.}\\
		\hline
	\end{tabular}
	\end{center}
	
	\begin{center}
	\begin{tabular}{| l | m{1.5cm} | l | m{4cm} | l | m{2.2cm} |}
		\hline
		\textbf{Id} & {RF19} & \textbf{Tipo de Usuario} & {Funcionario DISICO} & \textbf{Prioridad} & {Obligatorio}\\
		\hline
		\multicolumn{2}{|l|}{\textbf{Descripci�n}} & \multicolumn{4}{m{12cm}|}{Definir Items de configuraci�n del
		proyecto.}\\
		\hline
		\multicolumn{2}{|l|}{\textbf{Entrada}} & \multicolumn{4}{m{12cm}|}{Informaci�n del Item de Configuraci�n,
		acorde a lo descrito en la metodolog�a de SCM.}\\
		\hline
		\multicolumn{2}{|l|}{\textbf{Proceso}} & \multicolumn{4}{m{12cm}|}{Se registra el Item de Configuraci�n.}\\
		\hline
		\multicolumn{2}{|l|}{\textbf{Salida}} & \multicolumn{4}{m{12cm}|}{Pantalla de Items de Configuraci�n del proyecto.}\\
		\hline
	\end{tabular}
	\end{center}
	
	\begin{center}
	\begin{tabular}{| l | m{1.5cm} | l | m{4cm} | l | m{2.2cm} |}
		\hline
		\textbf{Id} & {RF20} & \textbf{Tipo de Usuario} & {Funcionario DISICO} & \textbf{Prioridad} & {Obligatorio}\\
		\hline
		\multicolumn{2}{|l|}{\textbf{Descripci�n}} & \multicolumn{4}{m{12cm}|}{Crear solicitud de cambio para alg�n proyecto.}\\
		\hline
		\multicolumn{2}{|l|}{\textbf{Entrada}} & \multicolumn{4}{m{12cm}|}{Datos de la solicitud, acorde a lo 
		definido por la metodolog�a de SCM.}\\
		\hline
		\multicolumn{2}{|l|}{\textbf{Proceso}} & \multicolumn{4}{m{12cm}|}{Se registra la solicitud de cambio.}\\
		\hline
		\multicolumn{2}{|l|}{\textbf{Salida}} & \multicolumn{4}{m{12cm}|}{Notificaci�n en pantalla del perfil del responsable 
		de analizar la solicitud.}\\
		\hline
	\end{tabular}
	\end{center}
	
	\begin{center}
	\begin{tabular}{| l | m{1.5cm} | l | m{4cm} | l | m{2.2cm} |}
		\hline
		\textbf{Id} & {RF21} & \textbf{Tipo de Usuario} & {Funcionario DISICO} & \textbf{Prioridad} & {Obligatorio}\\
		\hline
		\multicolumn{2}{|l|}{\textbf{Descripci�n}} & \multicolumn{4}{m{12cm}|}{Analizar solicitud de cambio.}\\
		\hline
		\multicolumn{2}{|l|}{\textbf{Entrada}} & \multicolumn{4}{m{12cm}|}{Informaci�n del impacto del cambio,
		acorde a lo definido por la metodolog�a de SCM.}\\
		\hline
		\multicolumn{2}{|l|}{\textbf{Proceso}} & \multicolumn{4}{m{12cm}|}{Se actualiza la solicitud.}\\
		\hline
		\multicolumn{2}{|l|}{\textbf{Salida}} & \multicolumn{4}{m{12cm}|}{Notificaci�n en el perfil del responsable
		de aprobar o  rechazar la solicitud.}\\
		\hline
	\end{tabular}
	\end{center}
	
	\begin{center}
	\begin{tabular}{| l | m{1.5cm} | l | m{4cm} | l | m{2.2cm} |}
		\hline
		\textbf{Id} & {RF22} & \textbf{Tipo de Usuario} & {Funcionario DISICO} & \textbf{Prioridad} & {Obligatorio}\\
		\hline
		\multicolumn{2}{|l|}{\textbf{Descripci�n}} & \multicolumn{4}{m{12cm}|}{Aprobar o rechazar solicitud.}\\
		\hline
		\multicolumn{2}{|l|}{\textbf{Entrada}} & \multicolumn{4}{m{12cm}|}{Selecci�n de la opci�n.}\\
		\hline
		\multicolumn{2}{|l|}{\textbf{Proceso}} & \multicolumn{4}{m{12cm}|}{Se actualiza la informaci�n
		de la solicitud.}\\
		\hline
		\multicolumn{2}{|l|}{\textbf{Salida}} & \multicolumn{4}{m{12cm}|}{Notificaci�n en el perfil del 
		responsable de SCM.}\\
		\hline
	\end{tabular}
	\end{center}
	
	\begin{center}
	\begin{tabular}{| l | m{1.5cm} | l | m{4cm} | l | m{2.2cm} |}
		\hline
		\textbf{Id} & {RF23} & \textbf{Tipo de Usuario} & {Funcionario DISICO} & \textbf{Prioridad} & {Obligatorio}\\
		\hline
		\multicolumn{2}{|l|}{\textbf{Descripci�n}} & \multicolumn{4}{m{12cm}|}{Completar formulario de 
		implementaci�n del cambio.}\\
		\hline
		\multicolumn{2}{|l|}{\textbf{Entrada}} & \multicolumn{4}{m{12cm}|}{Datos del formulario,
		acorde a lo definido por la metodolog�a de SCM.}\\
		\hline
		\multicolumn{2}{|l|}{\textbf{Proceso}} & \multicolumn{4}{m{12cm}|}{Registrar del formulario.}\\
		\hline
		\multicolumn{2}{|l|}{\textbf{Salida}} & \multicolumn{4}{m{12cm}|}{Pantalla de solicitudes de cambio.}\\
		\hline
	\end{tabular}
	\end{center}
	
	\begin{center}
	\begin{tabular}{| l | m{1.5cm} | l | m{4cm} | l | m{2.2cm} |}
		\hline
		\textbf{Id} & {RF24} & \textbf{Tipo de Usuario} & {Funcionario DISICO} & \textbf{Prioridad} & {Deseable}\\
		\hline
		\multicolumn{2}{|l|}{\textbf{Descripci�n}} & \multicolumn{4}{m{12cm}|}{Definir tareas de un proyecto.}\\
		\hline
		\multicolumn{2}{|l|}{\textbf{Entrada}} & \multicolumn{4}{m{12cm}|}{Datos de la tarea 
		(Proyecto, Descripci�n, Fecha programada).}\\
		\hline
		\multicolumn{2}{|l|}{\textbf{Proceso}} & \multicolumn{4}{m{12cm}|}{Registrar la tarea.}\\
		\hline
		\multicolumn{2}{|l|}{\textbf{Salida}} & \multicolumn{4}{m{12cm}|}{Pantalla de tareas del proyecto.}\\
		\hline
	\end{tabular}
	\end{center}
	
	\begin{center}
	\begin{tabular}{| l | m{1.5cm} | l | m{4cm} | l | m{2.2cm} |}
		\hline
		\textbf{Id} & {RF25} & \textbf{Tipo de Usuario} & {Funcionario DISICO} & \textbf{Prioridad} & {Deseable}\\
		\hline
		\multicolumn{2}{|l|}{\textbf{Descripci�n}} & \multicolumn{4}{m{12cm}|}{Actualizar tarea del proyecto.}\\
		\hline
		\multicolumn{2}{|l|}{\textbf{Entrada}} & \multicolumn{4}{m{12cm}|}{Nuevo estado de la tarea (Iniciada o Finalizada).}\\
		\hline
		\multicolumn{2}{|l|}{\textbf{Proceso}} & \multicolumn{4}{m{12cm}|}{Se actualiza la tarea.}\\
		\hline
		\multicolumn{2}{|l|}{\textbf{Salida}} & \multicolumn{4}{m{12cm}|}{Pantalla de Tareas del proyecto.}\\
		\hline
	\end{tabular}
	\end{center}
	
	\begin{center}
	\begin{tabular}{| l | m{1.5cm} | l | m{4cm} | l | m{2.2cm} |}
		\hline
		\textbf{Id} & {RF25} & \textbf{Tipo de Usuario} & {Jefe de �rea} & \textbf{Prioridad} & {Deseable}\\
		\hline
		\multicolumn{2}{|l|}{\textbf{Descripci�n}} & \multicolumn{4}{m{12cm}|}{Ver resumen del avance de los proyectos.}\\
		\hline
		\multicolumn{2}{|l|}{\textbf{Entrada}} & \multicolumn{4}{m{12cm}|}{Selecci�n de la opci�n.}\\
		\hline
		\multicolumn{2}{|l|}{\textbf{Proceso}} & \multicolumn{4}{m{12cm}|}{Se calcula el avance del proyecto en base a las
		fechas programadas y las tareas completadas.}\\
		\hline
		\multicolumn{2}{|l|}{\textbf{Salida}} & \multicolumn{4}{m{12cm}|}{Tabla y gr�ficos resumen de la informaci�n.}\\
		\hline
	\end{tabular}
	\end{center}


	\subsection{Requerimientos No Funcionales}
	\begin{center}
	\begin{tabular}{| c | m{13.2cm} |}
		\hline
			\textbf{Id} & \textbf{Descripci�n}\\
		\hline
			{RNF01} & {Permitir autenticaci�n de usuarios a trav�s de SSO.}\\
		\hline
			{RNF02} & {Utilizar la librer�a de componentes visuales de PrimeFaces 3.2.}\\
		\hline
			{RNF03} & {Utilizar el framework de persistencia Hibernate 4.1.}\\
		\hline
			{RNF04} & {Utilizar como servidor de aplicaciones Glassfish 3.1.1.}\\
		\hline
			{RNF05} & {Utilizar SQL Server 2008 R2 como sistema gestor de base de datos.}\\
		\hline
			{RNF06} & {La aplicaci�n debe ser f�cil de utilizar por personas con pocos 
			conocimientos en computaci�n. Tiempo de aprendizaje m�ximo 1 d�a.}\\
		\hline
			{RNF07} & {La aplicaci�n no debe verse afectada ante la falla de alg�n
			otro sistema de DISICO.}\\
		\hline
			{RNF08} & {La aplicaci�n debe estar disponible 24/7. Con un limite frontera aceptable
			de 20/7 para operaciones de correcci�n y mantenimiento }\\
		\hline
			{RNF09} & {La aplicaci�n debe soportar una concurrencia de 800 usuarios
			sin ver degradados los tiempos de respuestas.}\\
		\hline
			{RNF10} & {Los tiempos de respuesta del sistema deben ser de 8 segundos. 
			aceptando como limite 12 segundos solo en el caso de pantallas con resumen
			de informaci�n en tablas y gr�ficos}\\
		\hline
	\end{tabular}
	\end{center}

\section{Casos de Uso}

	\begin{table}[H]
	\begin{tabular}{|m{7.2cm}|m{7.2cm}|}
		\hline
		{\textbf{Nombre Caso de Uso}} & {Crear Solicitud de Requerimiento.}\\
		\hline
		{\textbf{Actores}} & {Solicitante.}\\
		\hline
		{\textbf{Prop�sito}} & {Permitir al solicitante el enviar
		solicitudes de requerimientos a alg�n �rea de DISICO.}\\
		\hline
		{\textbf{Resumen}} & {Este caso de uso comienza cuando el usuario
		desea enviar alguna solicitud de requerimiento a cualquiera de las �reas
		de DISICO, esto se hace ingresando la informaci�n de la solicitud y el 
		�rea a la que esta va dirigida.}\\
		\hline
		{\textbf{Tipo}} & {Esencial.}\\
		\hline
		{\textbf{Referencias Cruzadas}} & {}\\
		\hline
		{\textbf{Curso Normal (Usuario)}} & {\textbf{Curso Normal (Sistema)}}\\
		\hline
		{1. El usuario selecciona la opci�n Nueva Solicitud.} & {}\\
		{} & {2. El sistema despliega un formulario para el ingreso de los datos requeridos.}\\
		{3. El usuario completa la informaci�n de la solicitud.} & {}\\
		{4. El usuario selecciona el �rea a la que desea dirigir su solicitud.} & {}\\
		{5. El usuario env�a la solicitud.} & {}\\
		{} & {6. El sistema registra la solicitud.}\\
		{} & {7. El sistema env�a un correo electr�nico al usuario con el numero de su solicitud.}\\
		\hline
		{\textbf{Curso Alternativo (Usuario)}} & {\textbf{Curso Alternativo (Sistema)}}\\
		\hline
		{5. El usuario cancela la solicitud.} & {}\\
		{} & {6. El sistema vuelve al men� principal.}\\
		\hline
	\end{tabular}
	\caption{Caso de Uso Extendido de Crear Solicitud de Requerimiento}
	\end{table}
	
	\begin{table}[H]
	\begin{tabular}{|m{7.2cm}|m{7.2cm}|}
		\hline
		{\textbf{Nombre Caso de Uso}} & {Consultar Solicitud.}\\
		\hline
		{\textbf{Actores}} & {Solicitante.}\\
		\hline
		{\textbf{Prop�sito}} & {Permitir a un usuario consultar una solicitud a trav�s de 
		un numero de consulta.}\\
		\hline
		{\textbf{Resumen}} & {Este caso de uso comienza cuando un usuario desea 
		consultar el estado de una solicitud, esto puede hacerlo a trav�s del numero de consulta de 
		la solicitud, lo que le permitir� ver la solicitud aunque esta no haya sido enviada por el,
		o por fecha o asunto lo que le permitir� encontrar solo solicitudes enviadas por el mismo.}\\
		\hline
		{\textbf{Tipo}} & {Esencial.}\\
		\hline
		{\textbf{Referencias Cruzadas}} & {}\\
		\hline
		{\textbf{Curso Normal (Usuario)}} & {\textbf{Curso Normal (Sistema)}}\\
		\hline
		{1. El usuario ingresa la opci�n Consultar Solicitud.} & {} \\
		{} & {2. El sistema despliega una lista con todas las solicitudes del usuario
		ordenadas por fecha de la mas reciente a la menos reciente.} \\
		{3. El usuario tiene las siguientes opciones:} & {}\\
		{a. Ingresar un filtro de b�squeda: \textit{Ver secci�n Filtrar B�squeda}} & {}\\
		{b. Consultar a trav�s del numero de consulta: \textit{Ver secci�n Consultar 
		a trav�s de numero de consulta.}} & {}\\
		
	
		\hline
		{\textbf{Curso Alternativo (Usuario)}} & {\textbf{Curso Alternativo (Sistema)}}\\
		\hline
		{} & {}\\
		\hline
	\end{tabular}
		\caption{Caso de Uso Extendido Consultar Solicitud.}
	\end{table}
	
	\begin{center}
	\begin{tabular}{|m{7.2cm}|m{7.2cm}|}
		\hline
		\multicolumn{2}{|l|}{\textbf{Secci�n Filtrar B�squeda}} \\
		\hline
		{\textbf{Curso Normal (Usuario)}} & {\textbf{Curso Normal (Sistema)}}\\
		\hline
		{1. El usuario ingresa los filtros de b�squeda} & {}\\
		{} & {2. El sistema actualiza la lista de solicitudes desplegadas, mostrando solo las
		que coinciden con los filtros.}\\
		\hline
		{\textbf{Curso Alternativo (Usuario)}} & {\textbf{Curso Alternativo (Sistema)}}\\
		\hline
		{} & {}\\
		\hline
	\end{tabular}
	\end{center}
	
	\begin{center}
	\begin{tabular}{|m{7.2cm}|m{7.2cm}|}
		\hline
		\multicolumn{2}{|l|}{\textbf{Secci�n Consultar a trav�s de numero de consulta}} \\
		\hline
		{\textbf{Curso Normal (Usuario)}} & {\textbf{Curso Normal (Sistema)}}\\
		\hline
		{1. El usuario ingresa el numero de consulta y selecciona la opci�n consultar.} & {}\\
		{} & {2. El sistema encuentra la solicitud.}\\
		{} & {3. El sistema despliega la informaci�n de la solicitud.}\\
		{4. El usuario tiene la opci�n Comentar Solicitud: \textit{Consultar caso de uso Comentar Solicitud.}} & {}\\		\hline
		{\textbf{Curso Alternativo (Usuario)}} & {\textbf{Curso Alternativo (Sistema)}}\\
		\hline
		{} & {2. El sistema no encuentra la solicitud.}\\
		{} & {3. El sistema despliega un mensaje indicando que no se encontr� la solicitud.}\\
		\hline
	\end{tabular}
	\end{center}
	
	
	\begin{table}[H]
	\begin{tabular}{|m{7.2cm}|m{7.2cm}|}
		\hline
		{\textbf{Nombre Caso de Uso}} & {Comentar Solicitud.}\\
		\hline
		{\textbf{Actores}} & {Solicitante, Jefe de �rea, Funcionario DISICO.}\\
		\hline
		{\textbf{Prop�sito}} & {A�adir informaci�n, preguntas y respuestas a la solicitud.}\\
		\hline
		{\textbf{Resumen}} & {Este caso de uso comienza cuando un usuario 
		visualiza una solicitud y  selecciona la opci�n comentar y agrega un comentario.}\\
		\hline
		{\textbf{Tipo}} & {Esencial.}\\
		\hline
		{\textbf{Referencias Cruzadas}} & {}\\
		\hline
		{\textbf{Curso Normal (Usuario)}} & {\textbf{Curso Normal (Sistema)}}\\
		\hline
		{1. El usuario escoge la opci�n Comentar Solicitud.} & {}\\
		{} & {2. El sistema despliega el campo para el ingreso de texto.}\\
		{3. El usuario ingresa su comentario.} & {}\\
		{} & {4. El sistema registra el comentario.}\\
		{} & {5. El sistema notifica a los involucrados.}\\
		{\textbf{Curso Alternativo (Usuario)}} & {\textbf{Curso Alternativo (Sistema)}}\\
		\hline
		{3. El usuario cancela la acci�n.} & {}\\
		{} & {4. El sistema oculta el campo de texto.}\\
		\hline
	\end{tabular}
		\caption{Caso de Uso Extendido Comentar Solicitud.}
	\end{center}
	
	\begin{table}[H]
	\begin{tabular}{|m{7.2cm}|m{7.2cm}|}
		\hline
		{\textbf{Nombre Caso de Uso}} & {Gestionar Solicitud.}\\
		\hline
		{\textbf{Actores}} & {Jefe de �rea.}\\
		\hline
		{\textbf{Prop�sito}} & {Permitir al Jefe de �rea gestionar una solicitud que 
		ha sido enviada a su �rea.}\\
		\hline
		{\textbf{Resumen}} & {Este caso de uso comienza cuando el usuario desea 
		indicar que se debe hacer con una solicitud que ha sido enviada a su �rea, 
		para lo cual debe seleccionarla e indicar una opciones de gesti�n de solicitud, 
		permitiendo Asignar, Transferir, Rechazar, Convertir en Proyecto, Comentar, Responder.}\\
		\hline
		{\textbf{Tipo}} & {Esencial.}\\
		\hline
		{\textbf{Referencias Cruzadas}} & {}\\
		\hline
		{\textbf{Curso Normal (Usuario)}} & {\textbf{Curso Normal (Sistema)}}\\
		\hline
		{1. El usuario selecciona una solicitud.} & {}\\
		{} & {2. El sistema muestra el detalle de la solicitud.}\\
		{3. El usuario tiene las siguientes opciones:} & {}\\
		{a. Asignar Responsable:  \textit{Ver secci�n Asignar Responsable Solicitud} } & {}\\
		{b. Transferir Solicitud: \textit{Ver secci�n Transferir Solicitud} } & {}\\
		{c. Rechazar:  \textit{Ver secci�n Rechazar Solicitud} } & {}\\
		{d. Convertir en proyecto:  \textit{Ver secci�n Convertir Solicitud en Proyecto} } & {}\\
		{e. Comentar:  \textit{Ver Caso de Uso Comentar Solicitud} } & {}\\
		{f. Respuesta Directa:  \textit{Ver Caso de Uso Enviar Respuesta Directa.} } & {}\\
		{g. Respuesta Manual:  \textit{Ver secci�n Enviar Respuesta Manual} } & {}\\
		\hline
		{\textbf{Curso Alternativo (Usuario)}} & {\textbf{Curso Alternativo (Sistema)}}\\
		\hline
		{} & {}\\
		\hline
	\end{tabular}
		\caption{Caso de Uso Extendido Gestionar Solicitud.}
	\end{table}
	
	\begin{center}
	\begin{tabular}{|m{7.2cm}|m{7.2cm}|}
		\hline
		\multicolumn{2}{|l|}{\textbf{Secci�n Asignar Responsable Solicitud}} \\
		\hline
		{\textbf{Curso Normal (Usuario)}} & {\textbf{Curso Normal (Sistema)}}\\
		\hline
		{1. El usuario escoge la opci�n Asignar Responsable} & {}\\
		{} & {2. El sistema despliega un men� de selecci�n, con el nombre de
		todos los miembros del �rea.}\\
		{3. El usuario selecciona un responsable dentro de la lista.} & {}\\
		{4. El usuario presiona confirmar.} & {}\\
		{} & {5. El sistema registra la actualizaci�n y cambia el estado de la 
		solicitud a asignada.}\\
		{} & {6. El sistema muestra una notificaci�n en la pantalla del responsable.}\\
		\hline
		{\textbf{Curso Alternativo (Usuario)}} & {\textbf{Curso Alternativo (Sistema)}}\\
		\hline
		{4. El usuario presiona cancelar, se vuelve al paso 3.} & {}\\
		\hline
	\end{tabular}
	\end{center}
	
	\begin{center}
	\begin{tabular}{|m{7.2cm}|m{7.2cm}|}
		\hline
		\multicolumn{2}{|l|}{\textbf{Secci�n Transferir Solicitud}} \\
		\hline
		{\textbf{Curso Normal (Usuario)}} & {\textbf{Curso Normal (Sistema)}}\\
		\hline
		{1. El usuario escoge la opci�n Transferir Solicitud} & {}\\
		{} & {2. El sistema despliega una lista con el nombre de todas las �reas, 
		del departamento exceptuando el �rea actual.}\\
		{3. El usuario selecciona un �rea} & {}\\
		{} & {4. El sistema ingresar el motivo de la transferencia.}\\
		{5. El usuario ingresa el motivo de la transferencia y confirma la acci�n} & {}\\
		{} & {6. El sistema registra la transferencia y muestra notificaci�n en el
		perfil del Jefe de �rea al cual se transfiri� la solicitud.}\\
		{} & {7. El sistema vuelve a la pantalla principal.}\\
		\hline
		{\textbf{Curso Alternativo (Usuario)}} & {\textbf{Curso Alternativo (Sistema)}}\\
		\hline
		{} & {}\\
		\hline
	\end{tabular}
	\end{center}
	
	\begin{center}
	\begin{tabular}{|m{7.2cm}|m{7.2cm}|}
		\hline
		\multicolumn{2}{|l|}{\textbf{Secci�n Rechazar Solicitud}} \\
		\hline
		{\textbf{Curso Normal (Usuario)}} & {\textbf{Curso Normal (Sistema)}}\\
		\hline
		{1. El usuario selecciona la opci�n rechazar solicitud.} & {}\\
		{} & {2. El sistema solicita el ingreso del motivo del rechazo.}\\
		{3. El usuario ingresa el motivo del rechazo y confirma la acci�n.} & {}\\
		{} & {4. El sistema registra el motivo del rechazo y cambia el estado de la solicitud 
		a rechazada.}\\
		{} & {5. El sistema envi� un correo electr�nico al solicitante indicando el motivo
		por el cual su solicitud fue rechazada.}\\
		\hline
		{\textbf{Curso Alternativo (Usuario)}} & {\textbf{Curso Alternativo (Sistema)}}\\
		\hline
		{3. El usuario cancela la acci�n.} & {}\\
		{} & {4. El sistema vuelve al paso 2 del Caso de Uso Gestionar Solicitud. }\\
		\hline
	\end{tabular}
	\end{center}
	
	\begin{center}
	\begin{tabular}{|m{7.2cm}|m{7.2cm}|}
		\hline
		\multicolumn{2}{|l|}{\textbf{Secci�n Convertir Solicitud en Proyecto}} \\
		\hline
		{\textbf{Curso Normal (Usuario)}} & {\textbf{Curso Normal (Sistema)}}\\
		\hline
		{1. El usuario escoge la opci�n  Convertir en Proyecto} & {}\\
		{} & {2. El sistema solicitara confirmaci�n.}\\
		{3. El usuario confirma la acci�n} & {}\\
		{} & {4. El sistema modifica el estado de la solicitud a 
		convertida en proyecto.}
		{} & {5. El sistema da inicio al caso de uso Crear Proyecto}\\ 
		\hline
		{\textbf{Curso Alternativo (Usuario)}} & {\textbf{Curso Alternativo (Sistema)}}\\
		\hline
		{} & {}\\
		\hline
	\end{tabular}
	\end{center}
	
	\begin{center}
	\begin{tabular}{|m{7.2cm}|m{7.2cm}|}
		\hline
		\multicolumn{2}{|l|}{\textbf{Secci�n Enviar Respuesta Manual}} \\
		\hline
		{\textbf{Curso Normal (Usuario)}} & {\textbf{Curso Normal (Sistema)}}\\
		\hline
		{1. El usuario selecciona la opci�n de Respuesta Manual.} & {}\\
		{} & {2. El sistema solicita el ingreso de la direcci�n de correo
		de las personas a las que deber� ser respondida la solicitud.}\\
		{3. El usuario ingresa las direcciones de correo.} & {}\\
		{} & {4. El sistema solicita el ingreso del contenido del correo.}\\
		{5. El usuario ingresa el contenido del correo y confirma el env�o.} & {}\\
		{} & {6. El sistema env�a el correo a los destinatarios especificados.}\\
		\hline
		{\textbf{Curso Alternativo (Usuario)}} & {\textbf{Curso Alternativo (Sistema)}}\\
		\hline
		{} & {}\\
		\hline
	\end{tabular}
	\end{center}
	
	\begin{center}
	\begin{tabular}{|m{7.2cm}|m{7.2cm}|}
		\hline
		\multicolumn{2}{|l|}{\textbf{Secci�n }} \\
		\hline
		{\textbf{Curso Normal (Usuario)}} & {\textbf{Curso Normal (Sistema)}}\\
		\hline
		{} & {}\\
		{} & {}\\
		{} & {}\\
		\hline
		{\textbf{Curso Alternativo (Usuario)}} & {\textbf{Curso Alternativo (Sistema)}}\\
		\hline
		{} & {}\\
		\hline
	\end{tabular}
	\end{center}
	
	\begin{table}
	\begin{tabular}{|m{7.2cm}|m{7.2cm}|}
		\hline
		{\textbf{Nombre Caso de Uso}} & {.}\\
		\hline
		{\textbf{Actores}} & {.}\\
		\hline
		{\textbf{Prop�sito}} & {.}\\
		\hline
		{\textbf{Resumen}} & {.}\\
		\hline
		{\textbf{Tipo}} & {.}\\
		\hline
		{\textbf{Referencias Cruzadas}} & {.}\\
		\hline
		{\textbf{Curso Normal (Usuario)}} & {\textbf{Curso Normal (Sistema)}}\\
		\hline
		{} & {}\\
		\hline
		{\textbf{Curso Alternativo (Usuario)}} & {\textbf{Curso Alternativo (Sistema)}}\\
		\hline
		{} & {}\\
		\hline
	\end{tabular}
		\caption{Caso de Uso Extendido }
	\end{table}
	
	\begin{table}
	\begin{tabular}{|m{7.2cm}|m{7.2cm}|}
		\hline
		{\textbf{Nombre Caso de Uso}} & {Crear Proyecto.}\\
		\hline
		{\textbf{Actores}} & {Jefe de �rea.}\\
		\hline
		{\textbf{Prop�sito}} & {Crear un nuevo proyecto junto con la definici�n de sus
		involucrados.}\\
		\hline
		{\textbf{Resumen}} & {.}\\
		\hline
		{\textbf{Tipo}} & {.}\\
		\hline
		{\textbf{Referencias Cruzadas}} & {.}\\
		\hline
		{\textbf{Curso Normal (Usuario)}} & {\textbf{Curso Normal (Sistema)}}\\
		\hline
		{} & {}\\
		\hline
		{\textbf{Curso Alternativo (Usuario)}} & {\textbf{Curso Alternativo (Sistema)}}\\
		\hline
		{} & {}\\
		\hline
	\end{tabular}
		\caption{Caso de Uso Extendido }
	\end{table}
	
\section{Modelo Conceptual}
\chapter{Dise�o}
Explicar Dise�o
\section{Dise�o Arquitect�nico}
%	\subsection{Requerimientos Arquitecturales}
	Explicar que es arquitectura
	
	\subsection{Restricciones Arquitecturales}
	Antes de comenzar a definir la arquitectura del sistema es importante,
	especificar de forma explicita cuales son las restricciones arquitecturales
	que existen, las cuales se pueden deducir desde los requerimientos
	impuestos por el cliente, estas son:
	\begin{itemize}
		\item La aplicaci�n debe ser desarrollada en lenguaje de programaci�n Java.
		\item La aplicaci�n debe ser desarrollada siguiendo los est�ndar Java EE, 
		para mantener la compatibilidad con el servidor de aplicaciones del cliente
		(GlassFish).
		\item El acceso y guardado de los datos deben ser manejados con 
		persistencia, espec�ficamente haciendo uso del framework Hibernate.
		\item La autenticaci�n de usuario debe realizarse a trav�s de SSO, he 
		integrase con el sistema actual de login que hace uso de esta tecnolog�a.
	\end{itemize}
	
	\subsection{Estructura del Sistema}
	Para satisfacer los requerimientos y restricciones del cliente, se har� uso de la
	arquitectura Cliente-Servidor de 4-Capas la cual esta basada en el est�ndar 
	definido por Java EE 6.

	\begin{figure}[H]
		\begin{center}
			\includegraphics[scale=0.7]{imagenes/Arquitectura.png}
			\caption{Diagrama de Arquitectura}
		\end{center}
	\end{figure}	
	
	Las capas definidas son:
	\begin{enumerate}
		\item \textbf{Capa del Cliente:} Es la capa destinada a mostrar la interfaz 
		gr�fica de usuario y se ejecuta en el computador del cliente. Compuesta 
		principalmente por 2 partes: 
		\begin{itemize}
			\item Paginas web din�micas que contienen varios tipos de lenguajes de
			marcas (HTML, XML u otros), las cuales son generadas por la capa web.
			\item Un Navegador Web que interpreta las paginas enviadas por el 
			servidor.
		\end{itemize} 
		\item \textbf{Capa de Web:} Son componentes web creados con la
		tecnolog�a Java Server Faces, y contenidos en el contenedor web
		del servidor de aplicaciones Java EE. Los cuales se utilizan para generar
		las paginas web que son enviadas al cliente.
		\item \textbf{Capa de Negoci�:} Aqu� se define toda la l�gica
		particular del dominio del negoci�, a cargo de Enterprise Beans y Java
		Persistence Entities, estos reciben los datos que env�a el cliente, los 
		procesan de ser necesario y los env�an a la mecanismo de almacenamiento.
		Tambi�n recuperan informaci�n desde el mecanismo de almacenamiento, 
		la procesan de ser necesario y la env�an al cliente. Se encuentra 
		almacenados en el contenedor EJB del servidor de aplicaciones Java EE.
		
		\item \textbf{Capa de Datos:} Es un mecanismo de almacenamiento
		persistente, el cual almacena la informaci�n relevante para el sistema.
		En este caso el mecanismo es base de datos relacional.
	\end{enumerate}
	
	\subsection{Modelo de Control}
	

	\subsection{Estilo de Descomposici�n Modular}
	Para implementaci�n de la aplicaci�n es necesario utilizar un conjunto de 
	tecnolog�as Java que fueron definidas por el cliente, lo que nos sugiere que
	el estilo de descomposici�n mas adecuado y representativo es la descomposici�n
	Orientada a Objetos (OO). La cual nos permite estructurar el sistema como un
	conjunto de objetos d�bilmente acoplados.
	
\section{Dise�o L�gico}
	\subsection{Dise�o de Clases}
	\subsection{Patrones de Dise�o}
	
\section{Dise�o de Datos}
	\subsection{Modelo Entidad Relaci�n}
	\subsection{Diccionario de Datos}
\section{Dise�o de Interfaces}
\section{Dise�o de Pruebas}
	\subsection{Pruebas Unitarias}
	\subsection{Pruebas de Integraci�n}
	\subsection{Pruebas de }

%\appendix

%\include{appendix1}
%\include{appendix2}
%\include{appendix3}

\bibliographystyle{ieeetr}
\bibliography{template}

\end{document} 