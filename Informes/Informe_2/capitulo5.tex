\chapter{Dise�o}
Explicar Dise�o
\section{Dise�o Arquitect�nico}
%	\subsection{Requerimientos Arquitecturales}
	Explicar que es arquitectura
	
	\subsection{Restricciones Arquitecturales}
	Antes de comenzar a definir la arquitectura del sistema es importante,
	identificar de forma explicita cuales son las restricciones arquitecturales
	que existen, las cuales se pueden deducir desde los requerimientos
	impuestos por el cliente, estas restricciones son:
	\begin{itemize}
		\item La aplicaci�n debe ser desarrollada en lenguaje de programaci�n Java.
		\item La aplicaci�n debe ser desarrollada siguiendo los est�ndar Java EE, 
		para mantener la compatibilidad con el servidor de aplicaciones del cliente
		(GlassFish).
		\item El acceso y guardado de los datos deben ser manejados con 
		persistencia, espec�ficamente haciendo uso del framework Hibernate.
		\item La autenticaci�n de usuario debe realizarse a trav�s de SSO, he 
		integrase con el sistema actual de login que hace uso de esta tecnolog�a.
	\end{itemize}
	
	\subsection{Estructura del Sistema}
	Para satisfacer los requerimientos y restricciones del cliente, se utilizara la
	arquitectura Cliente-Servidor separada M�ltiples-Capas como lo plantea el est�ndar 
	definido por Java EE 6\cite{javaee}.

	\begin{figure}[H]
		\begin{center}
			\includegraphics[scale=0.7]{imagenes/Arquitectura.png}
			\caption{Diagrama de Arquitectura}
		\end{center}
	\end{figure}	
	
	Las capas definidas son:
	\begin{enumerate}
	%	\item \textbf{Capa del Cliente:} Es la capa destinada a mostrar la interfaz 
	%	gr�fica de usuario y se ejecuta en el computador del cliente. Compuesta 
	%	principalmente por 2 partes: 
	%	\begin{itemize}
	%		\item Paginas web din�micas que contienen varios tipos de lenguajes de
	%		marcas (HTML, XML u otros), las cuales son generadas por la capa web.
	%		\item Un Navegador Web que interpreta las paginas enviadas por el 
	%		servidor.
	%	\end{itemize} 
		
		\item \textbf{Capa Web:} Son componentes web creados con la
		tecnolog�a Java Server Faces, y contenidos dentro del contenedor web
		del servidor de aplicaciones Java EE. Los cuales se utilizan para generar
		las paginas web dinamicas, las cuales son enviadas en respuestas a las
		peticiones del clientes e interpretadas por el navegador web del mismo.
		
		\item \textbf{Capa de Negoci�:} Es la capa intermedia, la cual comunica la 	
		capa de datos con la capa web, esta capa contiene toda la l�gica
		particular del dominio del negoci� y tambi�n env�a y recupera informaci�n 
		desde la capa de Datos. Esta capa a su ves se compone por 
		dos sub-capas, las cuales se almacenan y trabajan conjuntamente
		dentro del contenedor EJB del servidor de aplicaciones Java EE. Estas 
		sub-capas son:
		\begin{itemize}
		 	\item \textbf{Capa EJB:} Esta capa contiene toda la l�gica
		 	de negocio y procesamiento de datos, y se se comunica con la capa 
		 	de persistencia para poder comunicarse con la capa de datos.
			\item \textbf{Capa de Persistencia:} Es capa contiene toda la l�gica
			para el manejo y uso de API de persistencia de Java 
			(en este caso implementada con hibernate), se encarga de mapear
			las tablas de la base de datos a entidades Java, el manejo de 
			transacciones y la inserci�n y recuperaci�n de datos desde esta.
		\end{itemize}
		
		\item \textbf{Capa de Datos:} Es un mecanismo de almacenamiento
		persistente, donde residen la informaci�n relevante para el sistema y es
		la encargada de acceder los mismos. En este caso particular esta formado
		por un gestor de base de datos relacional. Recibe solicitudes de 
		almacenamiento o recuperaci�n de informaci�n desde la capa de negocio.
	\end{enumerate}
	
	\subsection{Estilo de Descomposici�n Modular}
	En esta secci�n se describe como se descompone el sistema en diferentes modulos,
	esto se realizara utilizando el enfoque Orientado a Objetos (OO), para este caso 
	este enfoque es el mas adecuado, ya que el sistema debe ser implementado haciendo uso de 
	un lenguaje OO (particularmente java), as� se mantendr� la correcta correspondencia 
	entre los modelos y la implementaci�n.


	\subsection{Modelo de Control}
	

	
	
\section{Dise�o L�gico}
	\subsection{Dise�o de Clases}
	\subsection{Patrones de Dise�o}
	
\section{Dise�o de Datos}
	\subsection{Modelo Entidad Relaci�n}
	\subsection{Diccionario de Datos}
\section{Dise�o de Interfaces}
\section{Dise�o de Pruebas}
	\subsection{Pruebas Unitarias}
	\subsection{Pruebas de Integraci�n}
	\subsection{Pruebas de }