\chapter{Dise�o}
Explicar Dise�o
\section{Dise�o Arquitect�nico}
%	\subsection{Requerimientos Arquitecturales}
	Explicar que es arquitectura
	
	\subsection{Restricciones Arquitecturales}
	Antes de comenzar a definir la arquitectura del sistema es importante,
	identificar de forma explicita cuales son las restricciones arquitecturales
	que existen, las cuales se pueden deducir desde los requerimientos
	impuestos por el cliente, estas restricciones son:
	\begin{itemize}
		\item La aplicaci�n debe ser desarrollada en lenguaje de programaci�n Java.
		\item La aplicaci�n debe ser desarrollada siguiendo los est�ndar Java EE, 
		para mantener la compatibilidad con el servidor de aplicaciones del cliente
		(GlassFish).
		\item El acceso y guardado de los datos deben ser manejados con 
		persistencia, espec�ficamente haciendo uso del framework Hibernate.
		\item La autenticaci�n de usuario debe realizarse a trav�s de SSO, he 
		integrase con el sistema actual de login que hace uso de esta tecnolog�a.
	\end{itemize}
	
	\subsection{Estructura del Sistema}
	Para satisfacer los requerimientos y restricciones del cliente, se utilizara la
	arquitectura Cliente-Servidor separada 4-Capas como lo plantea el est�ndar 
	definido por Java EE 6\cite{javaee}.

	\begin{figure}[H]
		\begin{center}
			\includegraphics[scale=0.7]{imagenes/Arquitectura.png}
			\caption{Diagrama de Arquitectura}
		\end{center}
	\end{figure}	
	
	Las capas definidas son:
	\begin{enumerate}
		\item \textbf{Capa del Cliente:} Es la capa destinada a mostrar la interfaz 
		gr�fica de usuario y se ejecuta en el computador del cliente. Compuesta 
		principalmente por 2 partes: 
		\begin{itemize}
			\item Paginas web din�micas que contienen varios tipos de lenguajes de
			marcas (HTML, XML u otros), las cuales son generadas por la capa web.
			\item Un Navegador Web que interpreta las paginas enviadas por el 
			servidor.
		\end{itemize} 
		
		\item \textbf{Capa Web:} Son componentes web creados con la
		tecnolog�a Java Server Faces, y contenidos dentro del contenedor web
		del servidor de aplicaciones Java EE. Los cuales se utilizan para generar
		las paginas web que son enviadas al cliente.
		
		\item \textbf{Capa de Negoci�:} Aqu� se define toda la l�gica
		particular del dominio del negoci�, a cargo de Enterprise Beans y Java
		Persistence Entities, estos reciben los datos que env�a el cliente, los 
		procesan de ser necesario y los env�an a la mecanismo de almacenamiento.
		Tambi�n recuperan informaci�n desde el mecanismo de almacenamiento, 
		la procesan de ser necesario y la env�an al cliente. Los componentes 
		que conforman esta capa se encuentra almacenados 
		en el contenedor EJB del servidor de aplicaciones Java EE.
		
		\item \textbf{Capa de Datos:} Es un mecanismo de almacenamiento
		persistente, donde residen la informaci�n relevante para el sistema y es
		la encargada de acceder los mismos. En este caso particular esta formado
		por un gestor de base de datos relacional. Recibe solicitudes de 
		almacenamiento o recuperaci�n de informaci�n desde la capa de negocio.
	\end{enumerate}
	
	\subsection{Modelo de Control}
	

	\subsection{Estilo de Descomposici�n Modular}
	Para implementaci�n de la aplicaci�n es necesario utilizar un conjunto de 
	tecnolog�as Java que fueron definidas por el cliente, lo que nos sugiere que
	el estilo de descomposici�n mas adecuado y representativo es la descomposici�n
	Orientada a Objetos (OO). La cual nos permite estructurar el sistema como un
	conjunto de objetos d�bilmente acoplados.
	
\section{Dise�o L�gico}
	\subsection{Dise�o de Clases}
	\subsection{Patrones de Dise�o}
	
\section{Dise�o de Datos}
	\subsection{Modelo Entidad Relaci�n}
	\subsection{Diccionario de Datos}
\section{Dise�o de Interfaces}
\section{Dise�o de Pruebas}
	\subsection{Pruebas Unitarias}
	\subsection{Pruebas de Integraci�n}
	\subsection{Pruebas de }