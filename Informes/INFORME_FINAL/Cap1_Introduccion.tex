\chapter{Introducci\'on}

\label{intr}
La Direcci�n de Servicios de Informaci�n y Computaci�n (DISICO) \cite{disico}, con el objetivo de 
dar una mejor calidad de servicio, actualmente esta dividida en 3 �reas: �rea de Sistemas Financiero-Contables, 
�rea de Desarrollo y �rea de Redes, Comunicaciones y Soporte. Las funciones de las que DISICO 
es responsable se encuentran descritas en detalle en el Decreto 427 \cite{decreto427}, siendo algunas de
estas:
	\begin{itemize}
		\item Administrar todo el procesamiento de datos y documentaci�n, que por medio de sistemas 
		computacionales requiera la Universidad para su toma de decisiones.
		\item Establecer un catastro renovable en el tiempo de los requerimientos 
		inform�ticos de los usuarios de las distintas unidades de la Universidad.
		\item Interrelacionar los sistemas con las otras �reas de desarrollo de la 
		organizaci�n. 
		\item Mantener en constante actualizaci�n los sistemas de informaci�n y propender 
		a la creaci�n y desarrollo de sistemas en ambientes corporativos.
		\item Establecer pautas para obtener una estandarizaci�n en los sistemas computacionales.
	\end{itemize}

DISICO se encuentra en un proceso constante de crecimiento y mejora, para dar un mejor servicio a toda
la comunidad de la Universidad de Valpara�so. En esta b�squeda constante de aspectos a mejorar, es que 
se han realizado mejoras como por ejemplo, el Desarrollo de Metodolog�as para Software Quality Assurance (SQA)  
y Software Configuration Management (SCM) \cite{paloma}.
Pero a pesar de esto, se han detectado falencias tanto en los procesos de solicitudes de requerimientos, 
las cuales se realizan principalmente a trav�s del correo institucional y en las solicitudes de cambios, 
las que cuentan con la metodolog�a antes mencionada, pero aun no cuentan con una herramienta 
que permita su automatizaci�n.
\\\\
Por tanto se plantea dar soluci�n a dichas falencias y los problemas que estas producen
a trav�s del desarrollo de una plataforma que le permita automatizar los procesos existentes,
la cual permitir� un mejor control tanto del ciclo de vida de las tareas que se desarrollan en DISICO,
como de quienes las realizan.
 \\\\
Este documento se estructurara de la siguiente forma: 
\begin{itemize}
	\item \textbf{Capitulo 2:} Define el marco conceptual, donde se identifican
	 conceptos, t�rminos del dominio, junto con el an�lisis de las herramientas y 
	t�cnicas existentes.
	\item \textbf{Capitulo 3:} Aqu� se define la situaci�n actual, los problemas 
	de esta, la soluci�n propuesta y los objetivos que se esperan alcanzar en 
	este trabajo.
	\item \textbf{Capitulo 4:} En este capitulo se detalla todo el an�lisis del 
	sistema a construir, se identifican y definen requerimientos, casos de uso,
	diagramas de secuencia, de estados, y el modelo conceptual.
	\item \textbf{Capitulo 5:} En este capitulo se encuentra el dise�o de la aplicaci�n,
	esto incluye dise�o arquitectonico, l�gico, de datos, de interfaces de usuario y de pruebas.
	\item \textbf{Capitulo 6:} Este capitulo se describe las diferentes herramientas
	tanto de software como de hardware utilizadas para implementar la aplicaci�n.
	\item \textbf{Capitulo 7:} En este capitulo se encuentra la descripci�n de las pruebas 
	ejecutadas junto con el resultado obtenido de la ejecuci�n de estas.
	\item \textbf{Capitulo 8:} En este capitulo se encuentran descrito todos los pasos
	y configuraciones necesarias para la implantaci�n del sistema en un entorno de producci�n.
	\item \textbf{Capitulo 9:} Por ultimo se presentan las conclusiones obtenidas a lo largo de este
	trabajo.
\end{itemize}
	