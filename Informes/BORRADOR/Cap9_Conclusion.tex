\chapter{Conclusiones}
Luego de completar la fase de Pruebas, se puede establecer
que el sistema cumple con los objetivos propuestos, en este Trabajo de
Titulo. Por una parte permite la creaci�n y env�o de solicitudes de
requerimientos a DISICO y tambi�n permite el control, administraci�n y
asignaci�n de estos mismo por parte de los Jefes de cada �rea. 
Ayud�ndolos en la documentaci�n de los requerimientos y en 
la toma de decisiones en base a las diferentes m�tricas que el sistema
presenta.\\

Adem�s se ha implementado un modulo para la gesti�n
de los proyectos dentro de cada �rea, el cual permite tanto
documentar como monitorear el avance de los proyectos.
Este modulo ademas permite a los participante la asignaci�n y calendarizaci�n
de tareas sus propias tareas dentro de cada proyecto.\\

Por �ltimo tambi�n se implemento un modulo para la gesti�n
del cambio, que brinda soporte a la metodolog�a de SCM, 
propia de DISICO, eliminando de esta forma, el exceso de documentaci�n
en archivos Words aislados.\\

De esta forma se han integrado estos 3 conceptos (Solicitudes, Proyectos y Gesti�n 
del Cambio) dentro de una misma herramienta, que apoya gran parte
de las actividades de documentaci�n de DISICO, manteniendo todos mas 
ordenado y mejor organizado.\\

Por �ltimo despu�s de realizar las pruebas, se estableci� que se 
han cumplido con todos los requerimientos del cliente y que tambi�n
se a llegado a un nivel de aceptaci�n suficiente por parte de los usuarios finales.
Ya que en ning�n punto de las encuesta se encontraron valores de reprobaci�n
en ninguno de los conceptos evaluados. Y las observaciones realizadas por estos
est�n fuera de los requerimientos previamente definidos y son considerados
como mejoras a implementar, las cuales han sido incorporadas de a poco para lograr un nivel total de satisfacci�n por parte de los usuarios.\\

%TRABAJOS FUTUROS
Los trabajos futuros que se plantean son principalmente extensiones de la
herramienta implementada. Un trabajo importante seria incorporar medidas de
satisfacci�n de los usuarios que permitan conocer que tan conforme est�n los 
usuarios con las soluciones brindadas por parte de DISICO y sus funcionarios,
obteniendo as� un feedback por parte de los usuarios. Otro trabajo seria la 
implementaci�n de una versi�n de la herramienta para celulares, que le permita
tanto ha los jefes de �rea como funcionarios tener acceso a las principales 
funciones de atenci�n de solicitudes estando en terreno, reuniones, en sus hogares, etc. de forma simple y r�pida. Otro trabajo importante, seria extender 
el modulo que da soporte a la metodolog�a actual de SCM, para que pueda registrar interdependencia entre �tems de configuraci�n y facilitar as� la tarea de analizar el impacto de los cambios.