\chapter{Conclusi�n}
Luego de completar la fase de Pruebas, se puede establecer
que el sistema cumple con los objetivos propuestos, en este Trabajo de
Titulo. Por una parte permite la creaci�n y envio de solicitudes de
requerimientos a DISICO y tambi�n permite el control, administraci�n y
asignaci�n de estos mismo por parte de los Jefes de cada �rea. 
Ayudandolos en la documentaci�n de los requerimientos y en 
la toma de decisiones en base a las diferentes m�tricas que el sistema
presenta.\\

Ademas se ha implementado un modulo para la gesti�n
de los proyectos dentro de cada area, el cual permite tanto
documentar como monitorear el avance de los proyectos.
Este modulo ademas permite a los participante la asignaci�n y calendarizaci�n
de tareas sus propias tareas dentro de cada proyecto.\\

Por ultimo tambi�n se implemento un modulo para la gesti�n
del cambio, que brinda soporte a la metodolog�a de SCM, 
propia de DISICO, eliminando de esta forma, el exceso de documentaci�n
en archivos words aislados.\\

De esta forma se han integrado estos 3 conceptos (Solicitudes, Proyectos y Gesti�n 
del Cambio) dentro de una misma herramienta, que apoya gran parte
de las actividades de documentaci�n de DISICO, manteniendo todos mas 
ordenado y mejor organizado.\\

Por ultimo despues de realizar las pruebas, se estableci� que se 
han cumplido con todos los requerimientos del cliente y que tambi�n
se a llegado a un nivel de aceptaci�n suficiente por parte de los usuarios finales.
Ya que en ning�n punto de las encuesta se encontraron valores de reprobaci�n
en ninguno de los conceptos evaluados. Y las observaciones realizadas por estos
est�n fuera de los requerimientos previamente definidos y son considerados
como posibles mejoras a implementar en el futuro. Para lograr un
nivel total de satisfacci�n por parte de los usuarios.\\