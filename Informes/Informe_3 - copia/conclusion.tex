\chapter{Conclusi�n}
Luego de haber completado la fase de an�lisis y dise�o del sistema,
es posible concluir que su desarrollo es completamente factible, tanto
t�cnica como econ�micamente ya que se dispone de todos los recursos necesarios
y la aplicaci�n en si es no realizan ning�n procesamiento algor�tmico complejo.\\

Sin duda la soluci�n propuesta es la mejor alternativa frente a la falta 
de herramientas en el mercado que satisfagan las expectativas y 
requerimientos del cliente. Por lo que el dise�o de una herramienta a medida
es lo indicado. \\

Y claramente aunque tambi�n existe como siempre la posibilidad de dejar
las cosas como est�n, hacerlo no traer�a consigo ninguna mejora. Lo cual
es un punto clave que deber�n entender los usuarios, los cuales normalmente 
rechazan el cambio. Por tanto para que el sistema tenga �xito es importante
que presente nuevas opciones y mejoras respecto a los anteriores sistema, y
que estas sean percibidas f�cilmente por el usuario.\\

Siendo un poco mas cr�ticos, la aplicaci�n en si es bastante sencilla
y claro que existen varias mejoras que podr�an ser incorporadas, pero que por
factores de tiempo, se prefiri� dejar de lado moment�neamente
para poder implementar completamente las principales funcionalidades 
del sistema las cuales fueron especificadas por el cliente.\\
