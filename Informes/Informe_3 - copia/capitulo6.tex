\chapter{Pruebas}
	En este capitulo se detallan las pruebas realizadas, junto con los resultados 
	obtenidos durante la realizaci�n de estas.
	Las pruebas realizadas se dividen en:
	\begin{itemize}
		\item Pruebas Unitarias.
		\item Pruebas de Integraci�n.
		\item Pruebas de Rendimiento.
		\item Pruebas de Aceptaci�n.
		\item Pruebas Beta.
	\end{itemize}

	El principal enfoque de las pruebas es la detecci�n de errores.
	
	\section{Pruebas Unitarias}
	Las pruebas unitarias est�n enfocadas en probar unidades peque�as de c�digo. Para esto se
	dise�aron test automatizados, los cuales fueron desarrollados haciendo uso de la JUnit y de 
	Glassfih Embedded, dentro del cual se despliegan los EJB para ser utilizados durante la ejecuci�n de las
	pruebas.

	En la Tabla \ref{docTestUnit} se encuentran documentadas las pruebas realizadas junto con el
	prop�sito de cada una.
	
	\begin{table}[H]
	\begin{tabular}{|>{\columncolor[gray]{0.9}}m{3cm} | m{6cm}|m{5cm}|}
		\hline
		\rowcolor[gray]{0.7} {\textbf{Modulo}} & \textbf{Test} & \textbf{Proposito del Test}\\
		\hline
		\multirow{3}{*}{Resources} & {testGetValue} & {Verificar que el m�todo getValue es 
		capas de recuperar la cadena "ABCD" desde un archivo de propiedades.}\\
						\cline{2-3}
						& {testGetValueConEspacios} & {Verificar que el m�todo getValue es 
		capas de recuperar la cadena "A B C   D" desde un archivo de propiedades sin verse 
		afectado por la cantidad de espacios entre los caracteres.}\\
						\cline{2-3}
						& {testGetValueShort} & {Verificar que el m�todo getValueShort es 
						capas de recuperar cadena desde un archivo de propiedades y convertirla a Short siempre que cumpla 
						con el formato de este.}\\
						\cline{2-3}
						& {testGetValueShortNegativo} & {Verificar que el m�todo getValueShort es 
						capas de recuperar cadena desde un archivo de propiedades y convertirla a Short aunque este sea
						negativo.}\\
						\cline{2-3}
						& {testGetValueShortErrorEnString} & {Verificar que el m�todo getValueShort dispara la excepci�n
						NumberFormatException al leer un String desde el archivo de propiedades.}\\
						\cline{2-3}
						& {testGetValueShortErrorValorMayorAShort} & {Verificar que el m�todo getValueShort dispara la 		
						expeci�n NumberFormatException al leer un numero entero que excede el valor m�ximo de un Short.}\\
						\cline{2-3}
						& {testGetValueShortErrorValorDecimal} & {Verificar que el m�todo getValueShort dispara la excepci�n
						NumberFormatException al leer un valor con decimales desde el archivo de propiedades.}\\
						\cline{2-3}
						& {testGetValueInteger} & {Verificar que el m�todo getValueInteger es 
						capas de recuperar cadena desde un archivo de propiedades y convertirla a Integer siempre que cumpla 
						con el formato de este.}\\
	\end{tabular}
	\end{table}		
	
\begin{table}[H]
	\begin{tabular}{|>{\columncolor[gray]{0.9}}m{3cm} | m{6cm}|m{5cm}|}
		\hline
		\rowcolor[gray]{0.7} {\textbf{Modulo}} & \textbf{Test} & \textbf{Proposito del Test}\\		
		\hline
		\multirow{3}{*}{Resources} 
						& {testGetValueIntegerNegativo} & {Verificar que el m�todo getValueInteger es 
						capas de recuperar cadena desde un archivo de propiedades y convertirla a Integer aunque este sea
						negativo.}\\
						\cline{2-3}
						& {testGetValueIntegerErrorEnString} & {Verificar que el m�todo getValueInteger dispara la excepci�n
						NumberFormatException al leer un String desde el archivo de propiedades.}\\
						\cline{2-3}
						& {testGetValueIntegerErrorValorMayorAInteger} & {Verificar que el m�todo getValueInteger dispara la 	
						expeci�n NumberFormatException al leer un numero entero que excede el valor m�ximo de un Integer.}\\
						\cline{2-3}
						& {testGetValueIntegerErrorValorDecimal} & {Verificar que el m�todo getValueInteger dispara la excepci�n
						NumberFormatException al leer un valor con decimales desde el archivo de propiedades.}\\
						\cline{2-3}
						& {testGetValueLong} & {Verificar que el m�todo getValueLong es 
						capas de recuperar cadena desde un archivo de propiedades y convertirla a Long siempre que cumpla 
						con el formato de este.}\\
						\cline{2-3}
						& {testGetValueLongNegativo} & {Verificar que el m�todo getValueLong es 
						capas de recuperar cadena desde un archivo de propiedades y convertirla a Long aunque este sea
						negativo.}\\
						\cline{2-3}
						& {testGetValueLongErrorEnString} & {Verificar que el m�todo getValueLong dispara la excepci�n
						NumberFormatException al leer un String desde el archivo de propiedades.}\\
						\cline{2-3}
						& {testGetValueLongErrorValorMayorALong} & {Verificar que el m�todo getValueLong dispara la 		
						expeci�n NumberFormatException al leer un numero entero que excede el valor m�ximo de un Long.}\\
						\cline{2-3}
						& {testGetValueLongErrorValorDecimal} & {Verificar que el m�todo getValueLong dispara la excepci�n
						NumberFormatException al leer un valor con decimales desde el archivo de propiedades.}\\
						\cline{2-3}
						& {testGetPropertiesPath} & {}\\
						\cline{2-3}
						& {testGetPropertiesPathNotFound} & {}\\
						\cline{2-3}
						& {testGetPageList} & {}\\
						\cline{2-3}
						& {testGetMapPageList} & {}\\
						\cline{2-3}
						& {} & {}\\
						\cline{2-3}
						& {} & {}\\
						\cline{2-3}
						& {} & {}\\

		\hline
		\multirow{3}{*}{Modulo de Mensajeria} & {} & {}\\
						\cline{2-3}
						& {} & {}\\
						\cline{2-3}
						& {} & {}\\
						\cline{2-3}
						& {} & {}\\
						\cline{2-3}
						& {} & {}\\
						\cline{2-3}
						& {} & {}\\
						\cline{2-3}
						& {} & {}\\
						\cline{2-3}
						& {} & {}\\
						\cline{2-3}
						& {} & {}\\
		\hline
		\multirow{3}{*}{Modulo de Requerimientos} & {} & {}\\
						\cline{2-3}
						& {} & {}\\
						\cline{2-3}
						& {} & {}\\
						\cline{2-3}
						& {} & {}\\
						\cline{2-3}
						& {} & {}\\
						\cline{2-3}
						& {} & {}\\
						\cline{2-3}
						& {} & {}\\
						\cline{2-3}
						& {} & {}\\
						\cline{2-3}
						& {} & {}\\
		\hline
		\multirow{3}{*}{Modulo de Proyectos} & {} & {}\\
						\cline{2-3}
						& {} & {}\\
						\cline{2-3}
						& {} & {}\\
		\hline
		\multirow{3}{*}{Modulo de SCM} & {} & {}\\
						\cline{2-3}
						& {} & {}\\
						\cline{2-3}
						& {} & {}\\
		\hline
		\multirow{3}{*}{Modulo de Persistencia} & {} & {}\\
						\cline{2-3}
						& {} & {}\\
						\cline{2-3}
						& {} & {}\\

	\end{tabular}
		\caption{Test unitarios}
		\label{docTestUnit}
	\end{table}
